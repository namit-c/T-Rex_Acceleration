\documentclass[12pt]{article}

\usepackage{graphicx}
\usepackage{amsmath}
\usepackage{hhline}
\usepackage{booktabs}
\usepackage{multirow}
\usepackage{multicol}
\usepackage{tabularx}
\usepackage[normalem]{ulem}

\usepackage[left=0.70in, right=0.70in, top=0.70in, bottom=0.70in,
  headsep=0pt,% remove space between header and text body
  headheight=17pt% suggested by fancyhdr
  ]{geometry}

\usepackage{paralist}
\usepackage{amsfonts}


\usepackage{url}
\usepackage{longtable}
\usepackage{float}
\usepackage{soul}
\usepackage{xcolor}
\usepackage{ulem}

\oddsidemargin 0mm
\evensidemargin 0mm
\textwidth 160mm
\textheight 200mm
\renewcommand\baselinestretch{1.0}

\pagestyle {plain}
\pagenumbering{arabic}

\newcounter{stepnum}

\usepackage{color}

\newif\ifcomments\commentstrue

\ifcomments
\newcommand{\authornote}[3]{\textcolor{#1}{[#3 ---#2]}}
\newcommand{\todo}[1]{\textcolor{red}{[TODO: #1]}}
\else
\newcommand{\authornote}[3]{}
\newcommand{\todo}[1]{}
\fi

\newcommand{\wss}[1]{\authornote{blue}{SS}{#1}}

\title{SE 3XA3: Module Interface Specification T-Rex Acceleration}

\author{Team 15, Dev\textsuperscript{enthusiasts}
		\\ Zihao Du (duz12)
		\\ Andrew Balmakund (balmakua) 
		\\ Namit Chopra (choprn9)
}

\date{\today}

\begin {document}
 
\maketitle

\begin{table}[bp]
\caption{\bf Revision History}
\begin{tabularx}{\textwidth}{p{3cm}p{2cm}X}
\toprule {\bf Date} & {\bf Version} & {\bf Notes}\\
\midrule
03/18/2021 & 1.0 & Initial Draft\\
\textcolor{red}{04/07/2021} & \textcolor{red}{2.0} & \textcolor{red}{Revision 1}\\
\bottomrule
\end{tabularx}
\end{table}

\newpage

\section* {Model/Character}

\subsection* {Uses}
Pygame\\
Time

\subsection* {Syntax}
\subsubsection* {Exported Access Programs}

\resizebox{18cm}{!}{
\begin{tabular}{| l | l | l | l |}
\hline
\textbf{Routine name} & \textbf{In} & \textbf{Out} & \textbf{Exceptions}\\
\hline
Character & pygame.display, pygame.image & Character & IllegalArgumentException \\
\hline
\textcolor{red}{get\_rect} & \textcolor{red}{---} & \textcolor{red}{pygame.rect} & \textcolor{red}{---}\\
\hline
\textcolor{red}{get\_img} & \textcolor{red}{$\mathbb{Z}$} & \textcolor{red}{pygame.image} & \textcolor{red}{---}\\
\hline
set\_img & pygame.image \textcolor{red}{$\mathbb{R}$, $\mathbb{R}$} & --- & IllegalArgumentException\\
\hline
\textcolor{red}{set\_ducking\_img} & \textcolor{red}{---} & \textcolor{red}{pygame.image, $\mathbb{R}$, $\mathbb{R}$} & \textcolor{red}{IllegalArgumentException} \\
\hline
duck & pygame.image \sout{pygame.imge} & --- & IllegalArgumentException \\
\hline
\textcolor{red}{get\_ducking} & \textcolor{red}{---} & \textcolor{red}{$\mathbb{B}$} & \textcolor{red}{---}\\
\hline
stand & pygame.image, \sout{pygame.iamge} & --- & IllegalArgumentException\\
\hline
jump & pygame.image, \sout{pygame.image} & --- & IllegalArgumentException\\
\hline
\textcolor{red}{get\_jumping} & \textcolor{red}{---} & \textcolor{red}{$\mathbb{B}$} & \textcolor{red}{---}\\
\hline
\textcolor{red}{checkbounds} & \textcolor{red}{---} & \textcolor{red}{---} & \textcolor{red}{---}\\
\hline
invincible & \sout{pygame.image} & --- & \sout{IllegalArgumentException} \\
\hline
\textcolor{red}{get\_invincible} & \textcolor{red}{---} & \textcolor{red}{$\mathbb{B}$} & \textcolor{red}{---}\\
\hline
double\_jump\sout{ing} & --- & --- & ---\\
\hline
\textcolor{red}{get\_double\_jumping} & \textcolor{red}{---} & \textcolor{red}{$\mathbb{B}$} & \textcolor{red}{---}\\
\hline
slo\_mo & --- & --- & ---\\
\hline
\textcolor{red}{get\_slo\_mo} & \textcolor{red}{---} & \textcolor{red}{$\mathbb{B}$} & \textcolor{red}{---}\\
\hline
\textcolor{red}{is\_powered} & \textcolor{red}{---} & \textcolor{red}{$\mathbb{B}$} & \textcolor{red}{---}\\
\hline
update & pygame.image & --- & \sout{---} \textcolor{red}{IllegalArgumentExcption}\\
\hline
\textcolor{red}{get\_limit} & \textcolor{red}{---} & \textcolor{red}{$\mathbb{B}$} & \textcolor{red}{---}\\
\hline
\textcolor{red}{reset} & \textcolor{red}{pygam.image} & \textcolor{red}{---} & \textcolor{red}{---}\\
\hline
\textcolor{red}{get\_power\_time} & \textcolor{red}{---} & \textcolor{red}{$\mathbb{R}$} & \textcolor{red}{---}\\
\hline
\textcolor{red}{pause} & \textcolor{red}{---} & \textcolor{red}{---} & \textcolor{red}{---}\\
\hline
\textcolor{red}{resume} & \textcolor{red}{---} & \textcolor{red}{---} & \textcolor{red}{---}\\
\hline
\end{tabular}
}

\subsection* {Semantics}

\subsubsection* {State Variables}

\begin{tabular}{lll}
    game\_screen: pygame.display & // & Represents the screen \\
    img: \sout{pygame.image} \textcolor{red}{list of pygame.image} & // & Surface object with the image of the character\\
    \textcolor{red}{rect: pygame.rect} & \textcolor{red}{//} & \textcolor{red}{The rectangle of the character}\\
    is\_ducking: $\mathbb{B}$ & // & Is the character ducking\\
    is\_jumping: $\mathbb{B}$ & // & Is the character jumping\\
    is\_invincible: $\mathbb{B}$ & // & Is the character invincible\\
    is\_slo\_mo: $\mathbb{B}$ & // & Is the character slowing obstacles\\
    is\_double\_jumping: $\mathbb{B}$ & // & Can the character double jump\\
    movement: ($\mathbb{R,R}$) & // & horizontal and vertical speed of the character\\    \textcolor{red}{jumping\_limit: $\mathbb{Z}$} & \textcolor{red}{//} & \textcolor{red}{The number of jumps}\\
    \textcolor{red}{obtain\_powerup\_time: $\mathbb{R}$} & \textcolor{red}{//} & \textcolor{red}{The time when the character takes a powerup}\\
    \textcolor{red}{puase\_time: $\mathbb{R}$} & \textcolor{red}{//} & \textcolor{red}{Start time of a pause}\\
    \textcolor{red}{pause\_duration: $\mathbb{R}$} & \textcolor{red}{//} & \textcolor{red}{The duration of the pause}\\
    
\end{tabular}

\subsubsection* {Environment Variables}
None
\subsubsection* {State Invariant}

$is\_ducking \land is\_jumping = False$\\
$is\_invincible \land is\_slo\_mo \land is\_double\_jumping = False$\\
\textcolor{red}{$jumping\_limit < 3$}\\
\subsubsection* {Assumptions}

Constructor character is called first before other methods.

\subsubsection* {Access Routine Semantics}

\noindent Character(screen, char\_img):
\begin{itemize}
\item transition: game\_screen, img, movement, \textcolor{red}{jumping\_limit}. \textcolor{red}{rect} := screen, \sout{char\_img} \textcolor{red}{char\_img in NORMAL\_SIZE}, [0,0], \textcolor{red}{0}, \textcolor{red}{rectangle of char\_image}\\ All Boolean state variables are initialized to False\\
\textcolor{red}{All time state variables are initialized to 0}\\
\item output: \textit{out} := \textit{self}
\item exception: 
    \begin{itemize}[]
        \item $exc$ := $(img \equiv NULL) \Rightarrow IllegalArguementException$
        \item $exc$ := $(game\_screen \equiv NULL) \Rightarrow IllegalArguementError$
    \end{itemize}
\end{itemize}

\noindent \textcolor{red}{get\_rect():
\begin{itemize}
\item output: \textit{out} := \textit{rect}
\end{itemize}}

\noindent \textcolor{red}{get\_image(img\_number):
\begin{itemize}
\item output: \textit{out} := \textit{img[img\_number//IMAGE\_SELECTOR]}
\end{itemize}}

\noindent set\_img(new\_img, \textcolor{red}{ bottom}, \textcolor{red}{left}):
\begin{itemize}
    \item transition: img, \textcolor{red}{rect} := \sout{new\_img} \textcolor{red}{new\_img in NORMAL\_SIZE}, \textcolor{red}{rectangle of the new image with rect.bottom = bot; rect.left = left}
\item exception: 
    \begin{itemize}[]
        \item $exc$ := $(new\_img \equiv NULL) \Rightarrow IllegalArguementException$
    \end{itemize}    
\end{itemize}

\noindent \textcolor{red}{set\_ducking\_img(new\_img, bottom, left):
\begin{itemize}
    \item transition: img, rect :=  new\_img in DUCKING\_SIZE, rectangle of the new image with rect.bottom = bot; rect.left = left
\item exception: 
    \begin{itemize}[]
        \item $exc$ := $(new\_img \equiv NULL) \Rightarrow IllegalArguementException$
    \end{itemize}    
\end{itemize}}

\noindent duck(ducking\_img, \sout{inv\_ducking\_img} \textcolor{red}{char\_img}):
\begin{itemize}
\item transition: $\neg is\_jumping \Rightarrow (is\_ducking := False;$\\
\sout{$is\_invincible \Rightarrow img := inv\_ducking\_img \land \neg is\_invincible \Rightarrow img := ducking\_img)$}\\
\textcolor{red}{$img, rect := char\_img, \text{rectangle of char\_img})$}
\item exception:
    \begin{itemize}[]
        \item   $exc$ := $(any img \equiv NULL) \Rightarrow IllegalArguementException$
    \end{itemize}
\end{itemize}

\noindent \textcolor{red}{get\_ducking():
\begin{itemize}
\item output: \textit{out} := \textit{is\_ducking}
\end{itemize}}

\noindent stand(\sout{inv\_char}, char\_img):
\begin{itemize}
\item transition: $\neg is\_jumping \Rightarrow (is\_ducking := False;$\\
\sout{$is\_invincible \Rightarrow img := inv\_char \land \neg is\_invincible \Rightarrow img := char\_img)$}\\
\textcolor{red}{$set\_img(char\_img, screen\_rect.left - Y\_OFFSET, screen\_rect.bottom - X\_OFFSET)$}
\item exception:
    \begin{itemize}[]
        \item   $exc$ := $(char\_img \equiv NULL) \Rightarrow IllegalArguementException$
    \end{itemize}
\end{itemize}

\noindent jump(\sout{inv\_jumping\_char}, jumping\_img):
\begin{itemize}
\item transition: $\neg (is\_ducking \land is\_jumping) \Rightarrow (is\_jumping, movement[1], \textcolor{red}{jumping\_limit} := True, JUMPING\_SPEED, \textcolor{red}{jumping\_limit + 1} \land\\ \textcolor{red}{set\_img(char\_img, screen\_rect.left - Y\_OFFSET, screen\_rect.bottom - X\_OFFSET);}$\\
\textcolor{red}{$is\_double\_jumping \land is\_jumping \land jumping\_limit \leq DOUBLE\_JUMPING\_LIMIT \Rightarrow jumping\_limit, movement[1] := jumping\_limit + 1, DOUBLEJUMPING\_SPEED$}
\item exception: $exc$ := $(\_jumping\_img \equiv NULL) \Rightarrow IllegalArguementException$
\end{itemize}

\noindent \textcolor{red}{get\_jumping():
\begin{itemize}
\item output: \textit{out} := \textit{is\_jumping}
\end{itemize}}

\noindent \textcolor{red}{checkbounds():
\begin{itemize}
\item transition: $rect.bottom > screen\_rect.bottom - Y\_OFFSET \Rightarrow\\ rect.bottom, is\_jumping, jumping\_limit := screen\_rect.bottom - Y\_OFFSET, False, 0$
\end{itemize}}

\noindent invincible(\sout{inv\_char}):
\begin{itemize}
\item transition: $is\_invincible, $\sout{img,}$ is\_double\_jumping, is\_slo\_mo, \textcolor{red}{obtain\_powerup\_time, pause\_duration}\\ := True, $\sout{inv\_char,}$ False, False, \textcolor{red}{time(), 0}$\\
\item exception:
    \begin{itemize}[]
        \item   $exc$ := $(inv\_char \equiv NULL) \Rightarrow IllegalArguementException$
    \end{itemize}
\end{itemize}

\noindent \textcolor{red}{get\_invincible():
\begin{itemize}
\item output: \textit{out} := \textit{is\_invincible}
\end{itemize}}

\noindent double\_jump():
\begin{itemize}
\item transition: $is\_invincible, $\sout{img,}$ is\_double\_jumping, is\_slo\_mo, \textcolor{red}{obtain\_powerup\_time, pause\_duration}\\ := False, $\sout{inv\_char,}$ True, False, \textcolor{red}{time(), 0}$\\
\end{itemize}

\noindent \textcolor{red}{get\_double\_jump():
\begin{itemize}
\item output: \textit{out} := \textit{is\_double\_jumping}
\end{itemize}}

\noindent slo\_mo():
\begin{itemize}
\item transition: $is\_invincible, $\sout{img,}$ is\_double\_jumping, is\_slo\_mo, \textcolor{red}{obtain\_powerup\_time, pause\_duration}\\ := False, $\sout{inv\_char,}$ False, True, \textcolor{red}{time(), 0}$\\
\end{itemize}

\noindent \textcolor{red}{get\_slo\_mo():
\begin{itemize}
\item output: \textit{out} := \textit{is\_slo\_mo}
\end{itemize}}

\noindent \textcolor{red}{is\_powered():
\begin{itemize}
\item output: \textit{out} := \textit{$is\_double\_jumping \lor is\_invincible \lor is\_slo\_mo$}
\end{itemize}}

\noindent update(\textcolor{red}{char\_img}):
\begin{itemize}
\item transition: Change is\_double\_jumping or is\_invincible or is\_slo\_mo to False and the corresponding image back if the power up has already lasted for DURATION\_TIME. If is\_jumping is True, decrease movement[1] by GRAVITY, and check if the jumping ends(if the character back to ground) \textcolor{red}{If the character is not jumping or ducking, set the image to the original one}
\textcolor{red}{\item exception: $exc$ := $(char\_img \equiv NULL) \Rightarrow IllegalArguementException$}
\end{itemize}
\subsection*{Local Constants}
INIT\_SPEED = \sout{-20} \textcolor{red}{$\mathbb{Z}$}\\
GRAVITY = \sout{1}\textcolor{red}{$\mathbb{Z}$}  \\
DURATION = \sout{5}\textcolor{red}{$\mathbb{Z}$} \\
\textcolor{red}{NORMAL\_SIZE = \textcolor{red}{$(\mathbb{N,N})$}\\
DUCKING\_SIZE = $(\mathbb{N,N})$\\
X\_OFFSET = $\mathbb{Z}$\\
Y\_OFFSET = $\mathbb{Z}$\\
DOUBLEJUMPING\_SPEED = $\mathbb{N}$\\
IMAGE\_SELECTOR = $\mathbb{N}$\\
DOUBLE\_JUMPING\_LIMIT = $\mathbb{N}$\\
RESUME\_TIME = $\mathbb{N}$\\}
\medskip
%%%%%%%%%%%%%%%%%%%%%%%%%%%%%%%%%%%%%%%%%%%%%%%%%%%%%%%%%%%%%%%%%%
\newpage
%%%%%%%%%%%%%%%%%%%%%%%%%%%%%%%%%%%%%%%%%%%%%%%%%%%%%%%%%%%%%%%%%%

\section*{Model/Powerups}

\subsection* {Uses}
Pygame\\
Random\\
LoadAssets

\subsection* {Syntax}

\subsubsection* {Exported Access Programs}

\resizebox{17cm}{!}{
\begin{tabular}{| l | l | l | l |}
\hline
\textbf{Routine name} & \textbf{In} & \textbf{Out} & \textbf{Exceptions}\\
\hline
    Powerups & pygame.\sout{image}\textcolor{red}{display} Powerups, $\mathbb{N, N, R}$ & --- & IllegalArguementException\\
\hline
    \textcolor{red}{get\_rect} & \textcolor{red}{---} & \textcolor{red}{pygame.rect} & \textcolor{red}{---}\\
\hline
    get\_width & --- & $\mathbb{N}$ & ---\\
\hline
    set\_width & $\mathbb{N}$ & --- & IllegalArguementException\\
\hline
    get\_height & --- & $\mathbb{N}$ & ---\\
\hline
    set\_height & $\mathbb{N}$ & --- & IllegalArguementException\\
\hline
    get\_speed & --- & $\mathbb{R}$ & --- \\
\hline
    set\_speed & $\mathbb{N}$ & ---& --- \\
\hline
    get\_img & --- & \sout{string} \textcolor{red}{$\mathbb{Z}$} & ---\\
\hline
    set\_img & pygame.img & --- & IllegalArguementException\\
\hline
    get\_name & --- & String & --- \\
\hline
    \textcolor{red}{update} & \textcolor{red}{---} & \textcolor{red}{---} & \textcolor{red}{---}\\
\hline
\end{tabular}}

\subsection* {Semantics}

\subsubsection* {State Variables}
\begin{tabular}{lll}
\textcolor{red}{screen: pygame.display} & \textcolor{red}{//} & \textcolor{red}{Represents the game screen}\\
name: \sout{String \textcolor{red}{$\mathbb{Z}$}} & // & Represents the type of a power up\\
    width: $\mathbb{N}$ & // & Represents the width of a power up\\
    height: $\mathbb{N}$ & // & Represents the height of a power up\\
    speed: $\mathbb{R}$ & // & Represents the speed of a power up\\
    img: pygame.image & // & Surface object with a specified image drawn onto it\\
\textcolor{red}{rect: pygame.rect} & \textcolor{red}{//} & \textcolor{red}{Represents the rectangle shape of the powerup}\\
\textcolor{red}{screen\_rect: pygame.rect} & \textcolor{red}{//} & \textcolor{red}{Represents the game screen rectangle}\\
\end{tabular}

\subsubsection* {State Invariant}

\sout{None} \textcolor{red}{$name \in [0,TYPES]$}

\subsubsection* {Assumptions \& Design Decisions}

None

\subsubsection* {Access Routine Semantics}

\noindent Powerups(\sout{powerup\_img} \textcolor{red}{window}, width, height, speed):
\begin{itemize}
    \item transition: width, height, speed, \sout{img} \textcolor{red}{screen}, name, \textcolor{red}{img} := width, height, speed, \sout{powerup\_img,} \textcolor{red}{window, } random integer in [0,TYPES], \textcolor{red}{corresponding image with given width and height}
    \item output: $out := self$
\end{itemize}

\noindent \textcolor{red}{get\_rect():
\begin{itemize}
    \item output: $out$ := rect
\end{itemize}}

\noindent get\_width():
\begin{itemize}
    \item output: $out$ := width 
\end{itemize}

\noindent set\_width(new\_width):
\begin{itemize}
    \item transition: width := new\_width 
    \item exception: $exc$ := ($\text{new\_width} < 0$) $\Rightarrow  IllegalArgumentException$
\end{itemize}

\noindent get\_height():
\begin{itemize}
    \item output: $out$ := height
\end{itemize}


\noindent set\_height(new\_height):
\begin{itemize}
    \item transition: height := new\_height
    \item exception: $exc$ := ($\text{new\_height} < 0$) $\Rightarrow  IllegalArgumentException$
\end{itemize}


\noindent get\_speed():
\begin{itemize}
    \item output:$out$ := speed
\end{itemize}


\noindent set\_speed(new\_speed):
\begin{itemize}
    \item transition:speed := new\_speed
\end{itemize}


\noindent get\_img():
\begin{itemize}
    \item output: $out$ := img 
\end{itemize}


\noindent set\_img(new\_powerup\_img):
\begin{itemize}
    \item transition: img := new\_powerup\_img
    \item exception: $exc$ := $(new\_powerup\_img \equiv NULL) \Rightarrow IllegalArgumentException$
\end{itemize}

\noindent get\_name():
\begin{itemize}
    \item output: $out$ := name
\end{itemize}

\noindent \textcolor{red}{update():
\begin{itemize}
    \item transition: the object moves forward according to the speed
\end{itemize}}

\subsection*{Local Constants}
\textcolor{red}{Y\_OFFSET = $\mathbb{Z}$\\
TYPES = $\mathbb{Z}$}
%%%%%%%%%%%%%%%%%%%%%%%%%%%%%%%%%%%%%%%%%%%%%%%%%%%%%%%%%
\newpage
%%%%%%%%%%%%%%%%%%%%%%%%%%%%%%%%%%%%%%%%%%%%%%%%%%%%%%%%%

\section*{Model/Obstacle}

\subsection* {Uses}
Pygame


\subsection* {Syntax}

\subsubsection* {Exported Access Programs}

\begin{tabular}{| l | l | l | l |}
\hline
\textbf{Routine name} & \textbf{In} & \textbf{Out} & \textbf{Exceptions}\\
\hline
    Obstacle & String,$\mathbb{N, N, R}$, pygame.image & Obstacle & ---\\
\hline
    get\_width & --- & $\mathbb{N}$ & ---\\
\hline
    set\_width & $\mathbb{N}$ & --- & IllegalArgumentException\\
\hline
    get\_height & --- & $\mathbb{N}$ & ---\\
\hline
    set\_height & $\mathbb{N}$ & --- & IllegalArgumentException\\
\hline
    get\_speed & --- & $\mathbb{R}$ & --- \\
\hline
    set\_speed & $\mathbb{R}$ & ---& \textcolor{red}{IllegalArgumentException} \\
\hline
    get\_img & --- & pygame.image & ---\\
\hline
    set\_img & pygame.image & --- & IllegalArgumentException\\

\hline
    get\_rect & --- & ($\mathbb{R,R}$) & \\
\hline
    set\_rect & ($\mathbb{R,R}$) & --- & ---\\
\hline
\textcolor{red}{set\_rect} & \textcolor{red}{($\mathbb{R,R}$)} & \textcolor{red}{---} & \textcolor{red}{---} \\
\hline
\end{tabular}

\subsection* {Semantics}

\subsubsection* {State Variables}
\begin{tabular}{lll}
name: String & // & Represents the name of an obstacle\\
    width: $\mathbb{N}$ & // & Represents the width of an obstacle\\
    height: $\mathbb{N}$ & // & Represents the height of an obstacle\\
    speed: $\mathbb{R}$ & // & Represents the speed of an obstacle\\
    img: pygame.image & // & Surface object with a specified image drawn onto it\\
    rect: ($\mathbb{R,R}$) & // & X and Y coordinates of the obstacle\\
\end{tabular}

\subsubsection* {State Invariant}

None

\subsubsection* {Assumptions \& Design Decisions}

None

\subsubsection* {Access Routine Semantics}

\noindent Obstacle(name, width, height, speed, obstacle\_img):
\begin{itemize}
    \item transition:  name, width, height, speed, img, rect:= name, width, height, speed, obstacle\_img, obstacle\_img.get\_rect()
    \item output: $out := self$
\end{itemize}

\noindent get\_width():
\begin{itemize}
    \item output: $out$ := width 
\end{itemize}


\noindent set\_width(new\_width):
\begin{itemize}
    \item transition: width := new\_width 
    \item exception: $exc$ := ($\text{new\_width} < 0$) $\Rightarrow  IllegalArgumentException$
\end{itemize}

\noindent get\_height():
\begin{itemize}
    \item output: $out$ := height
\end{itemize}


\noindent set\_height(new\_height):
\begin{itemize}
    \item transition: height := new\_height
    \item exception: $exc$ := ($\text{new\_height} < 0$) $\Rightarrow  IllegalArgumentException$
\end{itemize}


\noindent get\_speed():
\begin{itemize}
    \item output: $out$ := speed
\end{itemize}


\noindent set\_speed(new\_speed):
\begin{itemize}
    \item transition: speed := new\_speed
    \item \textcolor{red}{exception: $exc$ := ($\text{new\_speed} < 0$) $\Rightarrow  IllegalArgumentException$}
\end{itemize}


\noindent get\_img():
\begin{itemize}
    \item output: $out$ := img 
\end{itemize}


\noindent set\_img(new\_osbtacle\_img):
\begin{itemize}
    \item transition: img := new\_osbtacle\_img
    \item exception: $exc$ := $(new\_osbtacle\_img \equiv NULL) \Rightarrow IllegalArgumentException $
\end{itemize}

\noindent get\_rect():
\begin{itemize}
    \item output: $out$ := rect 
\end{itemize}


\noindent set\_rect(x, y):
\begin{itemize}
    \item transition: rect.x, rect.y := x, y
\end{itemize}

\noindent \textcolor{red}{get\_pos(x, y):}
\begin{itemize}
    \item \textcolor{red}{output:  out:= rect.x, rect.y}
\end{itemize}


%%%%%%%%%%%%%%%%%%%%%%%%%%%%%%%%%%%%%%%%%%%%%%%%%%%%%%%%%
\newpage
%%%%%%%%%%%%%%%%%%%%%%%%%%%%%%%%%%%%%%%%%%%%%%%%%%%%%%%%%
\section*{Model/DetectCollision}

\subsection* {Uses}

Pygame

\subsection* {Syntax}

\subsubsection* {Exported Access Programs}
\resizebox{17cm}{!}{
\begin{tabular}{| l | l | l | l |}
\hline
\textbf{Routine name} & \textbf{In} & \textbf{Out} & \textbf{Exceptions}\\
\hline
    detect\_collision & pygame.sprite, seq of pygame.sprite & $\mathbb{B}$  & ---\\
\hline
    \sout{find\_collision} \textcolor{red}{find\_collision\_obstacle} & \sout{pygame.sprite, seq of pygame.sprite} \textcolor{red}{Character, seq of Obstacle} & Obstacle  & ---\\
\hline
    \textcolor{red}{find\_collision\_powerups} &  \textcolor{red}{Character, seq of Powerups} & Powerups  & ---\\
\hline
\end{tabular}
}

\subsection* {Semantics}

\subsubsection* {State Variables}

None

\subsubsection* {State Invariant}

None

\subsubsection* {Assumptions \& Design Decisions}

All pygame.sprite objects have been defined have been defined.

\subsubsection* {Access Routine Semantics}

detect\_collision(character, elements):
\begin{itemize}
    \item output: $out$ := ($\forall \text{ element} \in \text{elements } | \text{ (character.X}< \text{element.X} + \text{element.width}) \land (\text{character.X} + \text{character.width} > \text{element.X}) \land (\text{character.Y} < \text{element.Y} + \text{element.height}) \land (\text{character.Y} + \text{Character.height} > \text{element.Y})$)
\end{itemize}

\noindent \sout{find\_collision(character, elements):} \textcolor{red}{find\_collision\_obstacle(character, obstacles):}
\begin{itemize}
    \item output: $out$ := the \sout{element} \textcolor{red}{obstacle} in the sequence \sout{elements} \textcolor{red}{obstacles} that collides with the character.
\end{itemize}


\noindent  \textcolor{red}{find\_collision\_powerups(character, powerups):}
\begin{itemize}
    \item \textcolor{red}{output: $out$ := the powerup in the sequence powerups that collides with the character.}
\end{itemize}

%%%%%%%%%%%%%%%%%%%%%%%%%%%%%%%%%%%%%%%%%%%%%%%%%%%%%%%%%
\newpage
%%%%%%%%%%%%%%%%%%%%%%%%%%%%%%%%%%%%%%%%%%%%%%%%%%%%%%%%%
\section*{Model/UpdateEnvironment}

\subsection* {Uses}
Pygame \\
\textcolor{red}{Random}

\subsection* {Syntax}

\subsubsection* {Exported Access Programs}

\begin{tabular}{| l | l | l | l |}
\hline
\textbf{Routine name} & \textbf{In} & \textbf{Out} & \textbf{Exceptions}\\
\hline
    update\_floor & $\mathbb{Z}$, \textcolor{red}{$\mathbb{Z}$} & $\mathbb{Z}$ & ---\\
\hline
    update\_bg\_colour & $\mathbb{Z}$, $\mathbb{Z}$, seq of $\mathbb{Z}$ & seq of $\mathbb{Z}$ & ---\\
\hline
\end{tabular}

\subsection* {Semantics}

\subsubsection* {State Variables}

None

\subsubsection* {State Invariant}

None

\subsubsection* {Assumptions}

The game window and images are defined before any routines are called.

\subsubsection* {Access Routine Semantics}

update\_floor(floor\_position, \textcolor{red}{movement\_speed}):
\begin{itemize}
    \item output: \textit{out} := (floor\_position $\leq$ \textcolor{red}{BOUNDARY\_LEFT}) $\Rightarrow$ floor\_position = 0 $\lor$ (floor\_position = floor\_position - \sout{MOVEMENT\_SPEED} \textcolor{red}{movement\_speed})
\end{itemize}
\noindent update\_bg\_colour(current\_score, previous\_score, bg\_rgb):
\begin{itemize}
    \item output: \textit{out} := Every time the score is a multiple of \textcolor{red}{CHANGE\_BG\_INTERVAL}, the value of either red (bg\_rbg[\sout{0} \textcolor{red}{RED}]), green (bg\_rbg[\sout{1} \textcolor{red}{GREEN}]), or blue (bg\_rbg[\sout{2} \textcolor{red}{BLUE}]) changes, which is selected randomly. The new value of one of the field is: $(0 \leq x \leq 2 |$ $bg\_rgb[x] := (bg\_rgb[x]+$ \textcolor{red}{CHANGE\_BG\_VAL} $)\% $ \textcolor{red}{MAX\_RGB}). The updated bg\_rgb sequence is returned.
\end{itemize}
\subsubsection* {Local Constants}
\sout{MOVEMENT\_SPEED: 10} \\
CHANGE\_BG\textcolor{red}{\_VAL} = \sout{50} $\mathbb{N}$ \\
\textcolor{red}{MAX\_RGB = $\mathbb{N}$} \\
\textcolor{red}{CHANGE\_BG\_INTERVAL = $\mathbb{N}$} \\
\textcolor{red}{RED = $\mathbb{N}$} \\
\textcolor{red}{GREEN = $\mathbb{N}$} \\
\textcolor{red}{BLUE = $\mathbb{N}$} \\
\textcolor{red}{BOUNDARY\_LEFT = $\mathbb{Z}$}
%%%%%%%%%%%%%%%%%%%%%%%%%%%%%%%%%%%%%%%%%%%%%%%%%%%%%%%%%
\newpage
%%%%%%%%%%%%%%%%%%%%%%%%%%%%%%%%%%%%%%%%%%%%%%%%%%%%%%%%%
\section*{Model/Score}

\subsection* {Uses}

Time

\subsection* {Syntax}

\subsubsection* {Exported Access Programs}

\begin{tabular}{| l | l | l | l |}
\hline
\textbf{Routine name} & \textbf{In} & \textbf{Out} & \textbf{Exceptions}\\
\hline
    Score &--- & Score& ---\\
\hline 
    get\textcolor{red}{\_current}\_score & --- & $\mathbb{N}$ & --- \\
\hline
    update\_score & \textcolor{red}{time.time} & \textcolor{red}{$\mathbb{N}$, $\mathbb{N}$} & ---\\
\hline
    \sout{get\_high\_score} \textcolor{red}{get\_score} & --- & $\mathbb{N}$ & --- \\
\hline
    \textcolor{red}{reset\_score} & --- & --- & --- \\
\hline
    \textcolor{red}{boost} & --- & --- & --- \\
\hline
\end{tabular}

\subsection* {Semantics}

\subsubsection* {State Variables}

high\_score: $\mathbb{N}$ \\
current\_score: $\mathbb{N}$ \\
previous\_score: $\mathbb{N}$ \\
\sout{start\_time: $\mathbb{R}$} \\
\textcolor{red}{score\_boost = $\mathbb{N}$}

\subsubsection* {State Invariant}

None

\subsubsection* {Assumptions}

None

\subsubsection* {Access Routine Semantics}

Score():
\begin{itemize}
    \item output: $out$ := \textit{self} 
    \item transition: high\_score, current\_score, \sout{start\_time} \textcolor{red}{previous\_score, boost} := 0, 0, \sout{current time} \textcolor{red}{0, 0}
\end{itemize}
\noindent get\textcolor{red}{\_current}\_score():
\begin{itemize}
    \item output: $out$ := current\_score
\end{itemize}
\noindent update\_score(\textcolor{red}{start\_time}):
\begin{itemize}
    \item output: $out$ := current\_score, previous\_score
    \item transition: previous\_score, current\_score, high\_score := current\_score, ((current time - start\_time)$\cdot$ SCALE\_FACTOR \textcolor{red}{+ score\_boost $\cdot$ SCORE\_BOOST\_VAL}) rounded to the nearest natural number, (current\_score $>$ high\_score) $\Rightarrow$ current\_score \sout{$\lor$ high\_score}
\end{itemize}
\noindent get\_score():
\begin{itemize}
    \item output: $out$ := high\_score
\end{itemize}
\textcolor{red}{
\noindent reset\_score():
\begin{itemize}
    \item transition: current\_score, score\_boost := 0, 0
\end{itemize}
\noindent boost():
\begin{itemize}
    \item transition: score\_boost := score\_boost+1
\end{itemize}
}
\subsubsection* {Local Constants}
SCALE\_FACTOR = \sout{5} \textcolor{red}{$\mathbb{N}$} \\
\textcolor{red}{SCORE\_BOOST\_VAL = $\mathbb{N}$}

%%%%%%%%%%%%%%%%%%%%%%%%%%%%%%%%%%%%%%%%%%%%%%%%%%%%%%%%%
\newpage
%%%%%%%%%%%%%%%%%%%%%%%%%%%%%%%%%%%%%%%%%%%%%%%%%%%%%%%%%
\section*{\sout{Model/MainMenu}}

\subsection* {\sout{Uses}}
\sout{None}

\subsection* {\sout{Syntax}}

\subsubsection* {\sout{Exported Access Programs}}

\begin{tabular}{| l | l | l | l |}
\hline
\textbf{Routine name} & \textbf{In} & \textbf{Out} & \textbf{Exceptions}\\
\hline
    \sout{MainMenu} & --- & \sout{MainMenu} & ---\\
\hline
    \sout{change\_volume} & \sout{$\mathbb{N}$, $\mathbb{N}$} & --- & ---\\
\hline
    \sout{get\_volumes} & --- & \sout{$\mathbb{N}$, $\mathbb{N}$} & ---\\
\hline
\end{tabular}

\subsection* {\sout{Semantics}}

\subsubsection* {\sout{State Variables}}

\sout{background\_music\_volume: $\mathbb{N}$\\}
\sout{sound\_effects\_volume: $\mathbb{N}$}

\subsubsection* {\sout{State Invariant}}

\sout{None}

\subsubsection* {\sout{Assumptions}}

\sout{None}

\subsubsection* {\sout{Access Routine Semantics}}

\sout{MainMenu():}
\begin{itemize}
    \item \sout{output: $out$ := self}
    \item \sout{transition: background\_music\_volume, sound\_effects\_volume := MAX\_VOLUME, \\ MAX\_VOLUME}
\end{itemize}
\noindent \sout{change\_volume(new\_background\_volume, new\_sound\_effects\_volume):}
\begin{itemize}
    \item \sout{transition: background\_music\_volume, sound\_effects\_volume := new\_background\_volume, new\_sound\_effects\_volume}
\end{itemize}
\noindent \sout{get\_volumes():}
\begin{itemize}
    \item \sout{output: $out$ := background\_music\_volume, sound\_effects\_volume }
\end{itemize}
\subsection* {\sout{Local Constants}}
\sout{MAX\_VOLUME: 100}

%%%%%%%%%%%%%%%%%%%%%%%%%%%%%%%%%%%%%%%%%%%%%%%%%%%%%%%%%
\newpage
%%%%%%%%%%%%%%%%%%%%%%%%%%%%%%%%%%%%%%%%%%%%%%%%%%%%%%%%%

\section*{View/DisplayObstacle}

\subsection* {Uses}

Pygame\\
Time\\
Obstacle\\
Random\\
\textcolor{red}{DetectCollision}

\subsection* {Syntax}

\subsubsection* {Exported Access Programs}

\resizebox{17cm}{!}{
\begin{tabular}{| l | l | l | l |}
\hline
\textbf{Routine name} & \textbf{In} & \textbf{Out} & \textbf{Exceptions}\\
\hline
    DisplayObstacle & pygame.display & DisplayObstacle& IllegalArgumentException \\
\hline
    get\_obstacle\_list & --- & seq of Obstacle  & ---\\
\hline
    remove\_obstacle & Obstacle &  --- & ---\\
\hline
    generate\_obstacle &  \textcolor{red}{$\mathbb{R,R}$} seq of Obstacle, $\mathbb{R}$, \textcolor{red}{seq of Powerups} &  $\mathbb{R}$ & ---\\
\hline
    draw\_obstacle & $\mathbb{R,R}$, Obstacle &  --- & ---\\
\hline
    update\_obstacle\_display & --- &  --- & ---\\
\hline
 \textcolor{red}{update\_speed} & \textcolor{red}{$\mathbb{R}$} &  \textcolor{red}{---} & \textcolor{red}{IllegalArgumentException}\\
\hline
\end{tabular}
}

\subsection* {Semantics}

\subsubsection* {State Variables}

game\_screen: pygame.display \\
obtacle\_list: seq of Obstacle

\subsubsection* {State Invariant}

None

\subsubsection* {Assumptions}

None

\subsubsection* {Access Routine Semantics}

   DisplayObstacle(window):
\begin{itemize}
    \item output: \textit{out} := \textit{self}
    \item transition: game\_screen, obstacle\_list := window, []
    \item exception: \textit{exc}: (game\_screen = NULL $\Rightarrow$ IllegalArgumentExcpetion)
\end{itemize}
\noindent get\_obstacle\_list():
\begin{itemize}
    \item output: \textit{out} := obstacle\_list
\end{itemize}
\noindent remove\_obstacle(obstacle): 
\begin{itemize}
    \item transition: obstacle\_list := obstacle\_list - obstacle.
\end{itemize}
\noindent draw\_obstacle(current\_x, current\_y, obstacle): 
\begin{itemize}
    \item transition: game\_screen := Draw the given obstacle at a specific location (X, Y) coordinates using the \sout{obstacle image} \textcolor{red}{obstacle.get\_img()}.
\end{itemize}
\noindent generate\_obstacle(\textcolor{red}{current\_x, current\_y, }type\_of\_obstacles, prev\_obstacle\_spawn\_time, \textcolor{red}{generated\_powerups}):
\begin{itemize}
    \item transition: Generate a random kind of obstacle after APPROPRIATE\_TIME \sout{and} \textcolor{red}{,} add it to the obstacle\_list \textcolor{red}{ and draw it onto the screen}. \textcolor{red}{Also checks for any instance of a collision with an already generated powerups so that an obstacle and poweup don't overlap}.
    \item out: \textit{out} := current time
\end{itemize}
\noindent update\_obstacle\_display(): 
\begin{itemize}
    \item transition: For each obstacle in obstacle\_list, draw each obstacle at its new position by considering the speed of each object. Once the obstacle is outside the boundaries of the game\_screen, remove the obstacle from the list.
\end{itemize}

\noindent \textcolor{red}{update\_speed(new\_speed):}
\begin{itemize}
    \item \textcolor{red}{transition: For each obstacle in obstacle\_list, change the speed to new\_speed.}
\end{itemize}

\subsubsection* {Local Constants}
APPROPRIATE\_TIME = $\mathbb{Z}$
%%%%%%%%%%%%%%%%%%%%%%%%%%%%%%%%%%%%%%%%%%%%%%%%%%%%%%
\newpage 
%%%%%%%%%%%%%%%%%%%%%%%%%%%%%%%%%%%%%%%%%%%%%%%%%%%%%%%%%%%%%%%
\section*{View/DisplayPowerups}

\subsection* {Uses}

Pygame\\
Powerups\\
Random \\
\textcolor{red}{Time}\\
\textcolor{red}{DetectCollision}\\
\textcolor{red}{Obstacle}

\subsection* {Syntax}

\subsubsection* {Exported Access Programs}

\resizebox{15cm}{!}{
\begin{tabular}{| l | l | l | l |}
\hline
\textbf{Routine name} & \textbf{In} & \textbf{Out} & \textbf{Exceptions}\\
\hline
    DisplayPowerups & pygame.display & DisplayPowerups & IllegalArgumentException \\
\hline
    \textcolor{red}{get\_powerups\_list} & \textcolor{red}{---} & \textcolor{red}{---} & \textcolor{red}{---}\\
\hline
    \textcolor{red}{remove\_powerups\_list} & \textcolor{red}{Powerups} & \textcolor{red}{---} & \textcolor{red}{---}\\
\hline
    generate\_powerup & \textcolor{red}{$\mathbb{R}$, seq of Obstacles, $\mathbb{R}$} & --- & ---\\
\hline
    draw\_powerups & Powerups &  --- & ---\\
\hline
    update\_powerups & \textcolor{red}{seq of Obstacle} & --- & ---\\
\hline
    \textcolor{red}{update\_speed} & \textcolor{red}{$\mathbb{R}$} & --- & ---\\
\hline
\end{tabular}
}

\subsection* {Semantics}

\subsubsection* {State Variables}

powerups\_diplayed: seq of Powerups\\ 
game\_screen: pygame.display \\
\textcolor{red}{generate\_time: $\mathbb{R}$}

\subsubsection* {State Invariant}

None

\subsubsection* {Assumptions}

None

\subsubsection* {Access Routine Semantics}

display\_powerup(game\_screen):
\begin{itemize}
    \item output: \textit{out} := \textit{self}
    \item transition: game\_screen, powerups\_diplayed, \textcolor{red}{generate\_time} \\:= game\_screen, \sout{[]}, \textcolor{red}{pygame.sprite.Group(), time()}
    \item exception: \textit{exc}: (game\_screen = NULL $\Rightarrow$ IllegalArgumentExcpetion)
\end{itemize}

\noindent \textcolor{red}{get\_powerups\_list(): 
\begin{itemize}
    \item out: \textit{out} := powerups\_displayed
\end{itemize}}

\noindent \textcolor{red}{remove\_powerups\_list(p): 
\begin{itemize}
    \item transition: powerups\_displayed := powerups\_displayed.remove(p)
\end{itemize}}

\noindent generate\_powerup(\textcolor{red}{speed, obstacles, obstacle\_spawn\_time}):
\begin{itemize}
    \item transition: powerups\_diplayed := Add a random kind of powerup in the powerups\_displayed (a list of powerups displayed on the game\_screen) \textcolor{red}{, update generate\_time and draw it if there is no overlapping between obstacles and the new object and $time() - generate\_time > randint(RAND\_MIN,RAND\_MAX)\\ and time() - obstacle\_spawn\_time \geq INTERVAL\_TIME$}. 
\end{itemize}

\noindent draw\_powerups(p): 
\begin{itemize}
    \item transition: draw Powerup p on the screen
\end{itemize}

\noindent update\_powerups(\textcolor{red}{obstacles}):
\begin{itemize}
    \item transition: all elements in powerups\_displayed are updated by position and drawn. \textcolor{red}{If there is overlapping, remove the powerup, and if the powerup goes out of the screen , get rid of it.} 
\end{itemize}

\noindent \textcolor{red}{update\_speed(speed): 
\begin{itemize}
    \item transition: $\forall p \in powerups\_displayed: p.set\_speed(speed)$
\end{itemize}}
\subsection* {Local Constants}
\textcolor{red}{
POWERUPS\_WIDTH = $\mathbb{N}$\\
POWERUPS\_HEIGHT = $\mathbb{N}$\\
INTERVAL\_TIME = $\mathbb{N}$\\
RANDOMNESS = $\mathbb{R}$\\
RAND\_MIN = $\mathbb{N}$\\
RAND\_MAX = $\mathbb{N}$}
%%%%%%%%%%%%%%%%%%%%%%%%%%%%%%%%%%%%%%%%%%%%%%%%%%%%%%%%%
\newpage
%%%%%%%%%%%%%%%%%%%%%%%%%%%%%%%%%%%%%%%%%%%%%%%%%%%%%%%%%
\section*{View/DisplayEnvironment}

\subsection* {Uses}
\sout{UpdateEnvironment} \\
Pygame

\subsection* {Syntax}

\subsubsection* {Exported Access Programs}

\resizebox{17cm}{!}{
\begin{tabular}{| l | l | l | l |}
\hline
\textbf{Routine name} & \textbf{In} & \textbf{Out} & \textbf{Exceptions}\\
\hline
    DisplayEnvironment & pygame.display & DisplayEnvironment & IllegalArgumentException\\
\hline
    draw\_score & Score & --- & ---\\
\hline
    \sout{display\_instruction} \textcolor{red}{display\_msg} & String, \textcolor{red}{seq of $\mathbb{Z}$} & --- & ---\\
\hline
    draw\_floor & pygame.image, $\mathbb{Z}$ & --- & IllegalArgumentException\\
\hline
    draw\_background & pygame.image\textcolor{red}{, seq of $\mathbb{Z}$} & --- & IllegalArgumentException\\
\hline
    \textcolor{red}{display\_powerup} & \sout{pygame.image} \textcolor{red}{time.time} & --- & IllegalArgumentException \\
\hline
\end{tabular}
}
\subsection* {Semantics}

\subsubsection* {State Variables}

None

\subsubsection* {State Invariant}

game\_screen: pygame.display

\subsubsection* {Assumptions}

None

\subsubsection* {Access Routine Semantics}
\noindent DisplayEnvironment(window):
\begin{itemize}
    \item output: $out := self$
    \item transition: game\_screen := window
    \item exception: $exc$ := $(window \equiv NULL) \Rightarrow IllegalArgumentException $
\end{itemize}

\noindent draw\_score(score):
\begin{itemize}
    \item transition: The score is drawn on the game\_screen \textcolor{red}{using the display\_msg method.}
\end{itemize}

\noindent \sout{draw\_instruction (instructions)} \textcolor{red}{display\_msg(msg, msg\_pos)}:
\begin{itemize}
    \item transition: \sout{The instruction is shown on the screen for TIME and disappears after} \textcolor{red}{Draws msg on the game\_screen at the position msg\_pos}.
\end{itemize}

\noindent draw\_floor(floor, floor\_position):
\begin{itemize}
    \item transition : Draw the floor onto the game\_screen \textcolor{red}{at floor\_position}.
    \item exception: $exc$ := $(floor \equiv NULL) \Rightarrow IllegalArgumentException $
\end{itemize}

\noindent draw\_background(background\textcolor{red}{, bg\_rgb}):
\begin{itemize}
    \item transition : \textcolor{red}{Fills the game\_screen with the color bg\_rgb and draws} \sout{Draw the} background \sout{img} onto the game\_screen.
    \item exception: $exc$ := $(background \equiv NULL) \Rightarrow IllegalArgumentException $
\end{itemize}
\textcolor{red}{
\noindent draw\_powerup(time):
\begin{itemize}
    \item transition : Draws the time (remaining time left on acquired powerup) on the game\_screen.
\end{itemize}
}
\subsection*{\sout{Local Constants}}
\sout{TIME = 5 seconds}\\
%%%%%%%%%%%%%%%%%%%%%%%%%%%%%%%%%%%%%%%%%%%%%%%%%%%%%%%%%
\newpage
%%%%%%%%%%%%%%%%%%%%%%%%%%%%%%%%%%%%%%%%%%%%%%%%%%%%%%%%%
\section*{View/DisplayWindow}

\subsection* {Uses}
Pygame

\subsection* {Syntax}

\subsubsection* {Exported Access Programs}

\begin{tabular}{| l | l | l | l |}
\hline
\textbf{Routine name} & \textbf{In} & \textbf{Out} & \textbf{Exceptions}\\
\hline
    DisplayWindow & --- & DisplayWindow & ---\\
\hline
    get\_game\_screen &--- & pygame.display & ---\\
% \hline
%     MainMenu & --- & MainMenu & ---\\
\hline
\end{tabular}

\subsection* {Semantics}

\subsubsection* {State Variables}

game\_screen: pygame.display 

\subsubsection* {State Invariant}

None

\subsubsection* {Assumptions}

None

\subsubsection* {Access Routine Semantics}

DisplayWindow():
\begin{itemize}
    \item output: $out$ := $self$
    \item transition: game\_screen := a new pygame window with dimensions WIDTH $\times$ HEIGHT pixels
\end{itemize}
\noindent get\_game\_screen():
\begin{itemize}
    \item output: $out$ := game\_screen
\end{itemize}
\subsection* {Local Constants}
WIDTH = \sout{800} \textcolor{red}{$\mathbb{N}$} \\
HEIGHT = \sout{600} \textcolor{red}{$\mathbb{N}$}
%%%%%%%%%%%%%%%%%%%%%%%%%%%%%%%%%%%%%%%%%%%%%%%%%%%%%%%%%
\newpage
%%%%%%%%%%%%%%%%%%%%%%%%%%%%%%%%%%%%%%%%%%%%%%%%%%%%%%%%%
\section*{View/DisplayCharacter}

\subsection* {Uses}
Character
\subsection* {Syntax}

\subsubsection* {Exported Access Programs}

\begin{tabular}{| l | l | l | l |}
\hline
\textbf{Routine name} & \textbf{In} & \textbf{Out} & \textbf{Exceptions}\\
\hline
    DisplayCharacter & pygame.display, Character & --- & ---\\
\hline
    draw\_character & --- & --- & ---\\
\hline
\end{tabular}

\subsection* {Semantics}

\subsubsection* {State Variables}

game\_screen: pygame.display \\
game\_character: Character \\
\textcolor{red}{step: $\mathbb{Z}$}

\subsubsection* {State Invariant}

\sout{None}
\textcolor{red}{$0 \leq step \leq FRAME$}

\subsubsection* {Assumptions}

None

\subsubsection* {Access Routine Semantics}
DisplayCharacter(window, character):
\begin{itemize}
    \item transition: game\_screen, game\_character, \textcolor{red}{step} := window, character, \textcolor{red}{0} 
\end{itemize}
\noindent draw\_character():
\begin{itemize}
    \item transition: draw the character onto the game\_screen \textcolor{red}{according to step to mimic a gif.}\\
    \textcolor{red}{$step \leq FRAME \Rightarrow step := 0$; $step := step + 1$}
\end{itemize}
\subsection* {Local Constants}
\textcolor{red}{
FRAME = $\mathbb{N}$
}
%%%%%%%%%%%%%%%%%%%%%%%%%%%%%%%%%%%%%%%%%%%%%%%%%%%%%%%%%
\newpage
%%%%%%%%%%%%%%%%%%%%%%%%%%%%%%%%%%%%%%%%%%%%%%%%%%%%%%%%%
\section*{View/PlaySound}

\subsection* {Uses}
\sout{LoadAssets}\\
\textcolor{red}{Pygame}
\subsection* {Syntax}

\subsubsection* {Exported Access Programs}

\begin{tabular}{| l | l | l | l |}
\hline
\textbf{Routine name} & \textbf{In} & \textbf{Out} & \textbf{Exceptions}\\
\hline
    PlaySound & \textcolor{red}{seq of pygame.mixer.Sound} & PlaySound & ---\\
\hline
    \textcolor{red}{get\_sound\_effect} & \textcolor{red}{---} & \textcolor{red}{$\mathbb{R}$} & \textcolor{red}{---}\\
\hline
    \textcolor{red}{set\_sound\_effect} & \textcolor{red}{$\mathbb{R}$} & \textcolor{red}{---} & \textcolor{red}{IllegalArgumentException}\\
\hline
     \textcolor{red}{get\_background} & \textcolor{red}{---} & \textcolor{red}{$\mathbb{R}$} & \textcolor{red}{---}\\
\hline
    \textcolor{red}{set\_background} & \textcolor{red}{$\mathbb{R}$} & --- & \textcolor{red}{IllegalArgumentException}\\
\hline
    play\_bg\_music & --- & --- & ---\\
\hline
    play\_jump\_sound & --- & --- & ---\\
\hline
    play\_duck\_sound & --- & --- & ---\\
\hline
    play\_collision\_sound & --- & --- & ---\\
\hline
    play\_powerup\_sound & --- & --- & --- \\
\hline
    \textcolor{red}{play\_game\_over\_sound} & \textcolor{red}{---} & \textcolor{red}{---} & \textcolor{red}{---} \\
\hline
    \textcolor{red}{stop\_music} & \textcolor{red}{---} & \textcolor{red}{---} & \textcolor{red}{---}\\
\hline
\end{tabular}

\subsection* {Semantics}

\subsubsection* {State Variables}

background\_music: pygame.mixer\textcolor{red}{.Sound}\\
jump\_sound: pygame.mixer\textcolor{red}{.Sound}\\
duck\_sound: pygame.mixer\textcolor{red}{.Sound}\\
collision\_sound: pygame.mixer\textcolor{red}{.Sound}\\
powerup\_pickup\_sound: pygame.mixer\textcolor{red}{.Sound}\\
\textcolor{red}{game\_over\_sound: pygame.mixer.Sound}\\
\textcolor{red}{sound\_effect\_vol: $\mathbb{R}$}\\
\textcolor{red}{background\_vol: $\mathbb{R}$}
\subsubsection* {State Invariant}

None

\subsubsection* {Assumptions}

None

\subsubsection* {Access Routine Semantics}

PlaySound(sound\_list):
\begin{itemize}
    \item output: $out$ := $self$
    \item transition: background\_music, jump\_sound, duck\_sound, collision\_sound, powerup\_pickup\_sound, game\_over\_sound, sound\_effect\_vol, background\_VOL := the assets are loaded in from the LoadAssets module (for background, jump, duck, collision, powerup pickup and game over), SOUND\_EFFECT\_VOL\_INTIAL, BACKGROUND\_VOL\_INTIAL.
\end{itemize}

\noindent \textcolor{red}{get\_sound\_effect():}
\begin{itemize}
    \item \textcolor{red}{output: $out :=$ sound\_effect\_vol }
\end{itemize}

\noindent \textcolor{red}{set\_sound\_effect(new\_vol):}
\begin{itemize}
    \item \textcolor{red}{transition:   sound\_effect\_vol $:=$ new\_vol}
    \item \textcolor{red}{exception: $exc$ := $(0 < new\_vol \lor new\_vol > 1) \Rightarrow IllegalArgumentException $}
\end{itemize}

\noindent \textcolor{red}{get\_background():}
\begin{itemize}
    \item \textcolor{red}{output: $out :=$ background\_vol }
\end{itemize}

\noindent \textcolor{red}{set\_background(new\_vol):}
\begin{itemize}
    \item \textcolor{red}{transition:   background\_vol $:=$ new\_vol}
    \item \textcolor{red}{exception: $exc$ := $(0 < new\_vol \lor new\_vol > 1) \Rightarrow IllegalArgumentException $}
\end{itemize}


\noindent  play\_bg\_music():
\begin{itemize}
    \item transition: plays the background music \textcolor{red}{at background\_vol}.
\end{itemize}
\noindent play\_jump\_sound():
\begin{itemize}
    \item transition: plays the jump sound effect \textcolor{red}{at sound\_effect\_vol}.
\end{itemize}
\noindent  play\_duck\_sound():
\begin{itemize}
    \item transition: plays the duck sound effect \textcolor{red}{at sound\_effect\_vol}.
\end{itemize}
\noindent  play\_collision\_sound():
\begin{itemize}
    \item transition: plays the collision sound effect \textcolor{red}{at sound\_effect\_vol}.
\end{itemize}
\noindent  play\_powerup\_sound():
\begin{itemize}
    \item transition: plays the powerup pickup sound effect \textcolor{red}{at sound\_effect\_vol}.
\end{itemize}

\noindent  \textcolor{red}{play\_game\_over\_sound():}
\begin{itemize}
    \item \textcolor{red}{transition: plays the game over sound effect at sound\_effect\_vol}.
\end{itemize}

\noindent  \textcolor{red}{stop\_music():}
\begin{itemize}
    \item \textcolor{red}{transition: stops all current audio that is playing}.
\end{itemize}

\subsection* {Local Constants}
SOUND\_EFFECT\_VOL\_INITIAL = $\mathbb{R}$\\
BACKGROUND\_VOL\_INITIAL= $\mathbb{R}$
%%%%%%%%%%%%%%%%%%%%%%%%%%%%%%%%%%%%%%%%%%%%%%%%%%%%%%%%%
\newpage
%%%%%%%%%%%%%%%%%%%%%%%%%%%%%%%%%%%%%%%%%%%%%%%%%%%%%%%%%
\section*{View/DisplayMenu}

\subsection* {Uses}
\sout{MainMenu}\\
\textcolor{red}{Pygame}
\subsection* {Syntax}

\subsubsection* {Exported Access Programs}

\begin{tabular}{| l | l | l | l |}
\hline
\textbf{Routine name} & \textbf{In} & \textbf{Out} & \textbf{Exceptions}\\
\hline
    DisplayMenu & pygame.display & DisplayMenu & IllegalArgumentException\\
\hline
    display\_main\_menu & \textcolor{red}{pygame.image} & --- & ---\\
\hline
    display\_pause\_menu &  \textcolor{red}{pygame.image} & --- & ---\\
\hline
    \sout{display\_exit\_menu} \textcolor{red}{display\_end\_menu} & \textcolor{red}{pygame.image} & --- & ---\\
\hline
     display\_setting\_menu & \textcolor{red}{pygame.image} & --- & ---\\
\hline
   \textcolor{red}{display\_resume\_menu} & \textcolor{red}{$\mathbb{N}$} & \textcolor{red}{---} & \textcolor{red}{---} \\
\hline
   \textcolor{red}{display\_instruction\_menu} & \textcolor{red}{pygame.image} & \textcolor{red}{---} & \textcolor{red}{---} \\
\hline
\end{tabular}

\subsection* {Semantics}

\subsubsection* {State Variables}

game\_screen: pygame.display

\subsubsection* {State Invariant}

None

\subsubsection* {Assumptions}

None

\subsubsection* {Access Routine Semantics}

DisplayMenu(window):
\begin{itemize}
    \item output: $out$ := $self$
    \item transition: game\_screen := window
    \item exception: $exc$ := $(window \equiv NULL) \Rightarrow IllegalArgumentException $
\end{itemize}

\noindent display\_main\_menu(\textcolor{red}{main\_menu\_img}):
\begin{itemize}
    \item transition: game\_screen := \sout{Main menu} \textcolor{red}{Display the main menu image which contains layout for} \sout{with} `Play', `Quit' and `Setting' \sout{buttons}. \textcolor{red}{The image will also display the keybindings that correspond to accessing each menu. }
\end{itemize}

\noindent display\_pause\_menu(\textcolor{red}{pause\_menu\_img}):
\begin{itemize}
    \item transition: game\_screen := \sout{Pause menu} \textcolor{red}{Display the pause menu image which contains} \sout{with} `Resume' \sout{button} to resume back to the current game or `Exit' \sout{button} to go to the main menu. \textcolor{red}{The image will also display the keybindings that correspond to accessing each menu. }
\end{itemize}

\noindent \sout{display\_exit\_menu} \textcolor{red}{display\_end\_menu}(\textcolor{red}{exit\_menu\_img}):
\begin{itemize}
    \item transition: game\_screen := \sout{Exit menu} \textcolor{red}{Display the exit menu image which contains} \sout{with} `Return` \sout{button} to the main menu after game session has ended \textcolor{red}{ and `Quit` to quit the game application (terminating the game)}. \textcolor{red}{The image will also display the keybindings that correspond to accessing each menu. }
\end{itemize}

\noindent display\_setting\_menu(\textcolor{red}{setting\_menu\_img}):
\begin{itemize}
    \item transition: game\_screen := \sout{Setting menu that can be used to change the volume and theme.} \textcolor{red}{Display the setting menu image and also contains buttons which are plus and minus to change the background volume and sound effect volume. The img will also contain a `Back` to return to the main menu and `Confirm` to save the current changes of background and sound effect volume setting}
\end{itemize}

\noindent \textcolor{red}{display\_resume\_menu(time\_remaining):}
\begin{itemize}
    \item \textcolor{red}{transition: game\_screen := Displays the current time\_ remaining on the screen.}
\end{itemize}

\noindent \textcolor{red}{display\_instruction\_menu(instruction\_menu\_img):}
\begin{itemize}
    \item \textcolor{red}{transition: game\_screen := Displays the instruction menu image on the screen which contains information about the game and how to play it. It also contains information about the keybinding to return back to the main menu.}
\end{itemize}
%%%%%%%%%%%%%%%%%%%%%%%%%%%%%%%%%%%%%%%%%%%%%%%%%%%%%%%%%
\newpage
%%%%%%%%%%%%%%%%%%%%%%%%%%%%%%%%%%%%%%%%%%%%%%%%%%%%%%%%%
\section*{View/LoadAssets}

\subsection* {Uses}

Pygame


\subsection* {Syntax}

\subsubsection* {Exported Access Programs}

\begin{tabular}{| l | l | l | l |}
\hline
\textbf{Routine name} & \textbf{In} & \textbf{Out} & \textbf{Exceptions}\\
\hline
    load\_floor & --- & pygame.image & IllegalArugementException\\
\hline
    load\_background & --- & pygame.image & IllegalArugementException\\
\hline
    load\_character & --- & seq of pygame.image & IllegalArugementException\\
\hline
load\_all\_obstacle & --- & seq of pygame.image & IllegalArugementException\\
\hline
load\_all\_powerups & --- &  seq of pygame.image & IllegalArugementException\\
\hline
load\_main\_menu & --- &   pygame.image & IllegalArugementException\\
\hline
\textcolor{red}{load\_pause\_menu} & --- & \textcolor{red}{pygame.image} & \textcolor{red}{IllegalArugementException}\\
\hline
\textcolor{red}{load\_end\_menu} & --- & \textcolor{red}{pygame.image} & \textcolor{red}{IllegalArugementException}\\
\hline
\textcolor{red}{load\_setting\_menu} & --- & \textcolor{red}{pygame.image} & \textcolor{red}{IllegalArugementException}\\
\hline
\textcolor{red}{load\_instruction\_menu} & --- & \textcolor{red}{pygame.image} & \textcolor{red}{IllegalArugementException}\\
\hline
load\_sound & --- &   pygame.mixer & IllegalArugementException\\
\hline
\end{tabular}

\subsection* {Semantics}

\subsubsection* {State Variables}

None

\subsubsection* {State Invariant}

None

\subsubsection* {Assumptions}

All the files are in the appropriate directory with proper names and format.

\subsubsection* {Access Routine Semantics}

load\_floor():
\begin{itemize}
    \item output: \textit{out} := returns a Pygame image object with the floor image loaded in.
    \item exception: \textit{exc} := $(\text{FLOOR\_IMG} \equiv \neg FileExists) \Rightarrow FileNotFoundError$
\end{itemize}

\noindent load\_background():
\begin{itemize}
    \item output: \textit{out} := returns a Pygame image object with the background image loaded in.
    \item exception: \textit{exc} := $(\text{BACKGROUND\_IMG} \equiv \neg FileExists) \Rightarrow FileNotFoundError$
\end{itemize}

\noindent load\_character():
\begin{itemize}
    \item output: \textit{out} := returns a sequence of Pygame image objects with images of \sout{different characters} \textcolor{red}{the character performing different actions (including jumping, ducking, and running)} loaded in.
    \item exception: \textit{exc} := $(\text{CHARACTER\_IMG} \equiv \neg FileExists) \Rightarrow FileNotFoundError$
\end{itemize}

\noindent load\_all\_obstacle\textcolor{red}{s}():
\begin{itemize}
    \item output: \textit{out} := returns a sequence of Pygame image objects with all obstacle images loaded in.
    \item exception: \textit{exc} := $(\text{OBSTACLE\_IMG} \equiv \neg FileExists) \Rightarrow FileNotFoundError$
\end{itemize}

\noindent load\_all\_powerups():
\begin{itemize}
    \item output: \textit{out} := returns a \textcolor{red}{sequence of }Pygame image object\textcolor{red}{s} with the all powerup images loaded in.
    \item exception: \textit{exc} := $(\text{POWERUP\_IMG} \equiv \neg FileExists) \Rightarrow FileNotFoundError$
\end{itemize}

\noindent load\_main\_menu():
\begin{itemize}
    \item output: \textit{out} := returns a Pygame image object with the main menu image loaded in.
    \item exception: \textit{exc} := $(\text{MAINMENU\_IMG} \equiv \neg FileExists) \Rightarrow FileNotFoundError$
\end{itemize}

\noindent \textcolor{red}{load\_pause\_menu():
\begin{itemize}
    \item output: \textit{out} := returns a Pygame image object with the pause menu image loaded in.
    \item exception: \textit{exc} := $(\text{MAINMENU\_IMG} \equiv \neg FileExists) \Rightarrow FileNotFoundError$
\end{itemize}
}

\noindent \textcolor{red}{load\_end\_menu():
\begin{itemize}
    \item output: \textit{out} := returns a Pygame image object with the end menu image loaded in.
    \item exception: \textit{exc} := $(\text{MAINMENU\_IMG} \equiv \neg FileExists) \Rightarrow FileNotFoundError$
\end{itemize}
}

\noindent \textcolor{red}{load\_setting\_menu():
\begin{itemize}
    \item output: \textit{out} := returns a Pygame image object with the setting menu image loaded in.
    \item exception: \textit{exc} := $(\text{MAINMENU\_IMG} \equiv \neg FileExists) \Rightarrow FileNotFoundError$
\end{itemize}
}

\noindent \textcolor{red}{load\_instruction\_menu():
\begin{itemize}
    \item output: \textit{out} := returns a Pygame image object with the instruction menu image loaded in.
    \item exception: \textit{exc} := $(\text{MAINMENU\_IMG} \equiv \neg FileExists) \Rightarrow FileNotFoundError$
\end{itemize}
}

\noindent load\_sound():
\begin{itemize}
    \item output: \textit{out} := returns Pygame sound objects with the background music and sound effects loaded in.
    \item exception: \textit{exc} := $(\text{SOUND\_MP3} \equiv \neg FileExists) \Rightarrow FileNotFoundError$
\end{itemize}
\subsection*{Local Constants}
FLOOR\_IMG = \sout{`floor.png'} \textcolor{red}{image} \\
BACKGROUND\_IMG = \sout{`background.png'} \textcolor{red}{image} \\
CHARACTER\_IMG = \sout{[`character.png', `character\_invisible.png', 'character\_slomo.png']} \textcolor{red}{seq of image}\\
OBSTACLE\_IMG = \sout{[`obstacle1.png', `obstacle2.png']} \textcolor{red}{seq of image}\\
POWERUP\_IMG = \sout{[`powerup1.png', `powerup2.png', `powerup3.png', `powerup4.png']} \textcolor{red}{seq of image}\\
MAINMENU\_IMG = \sout{`mainmenu.png'} \textcolor{red}{image}\\ 
SOUND\_MP3 = \sout{[`sound1.mp3', `sound2.mp3', `sound3.mp3', `sound4.mp3', `sound5.mp3']} \textcolor{red}{seq of audio}\\

\medskip

%%%%%%%%%%%%%%%%%%%%%%%%%%%%%%%%%%%%%%%%%%%%%%%%%%%%%%%%%
\newpage
%%%%%%%%%%%%%%%%%%%%%%%%%%%%%%%%%%%%%%%%%%%%%%%%%%%%%%%%%
\section*{Controller/GameController}

\subsection* {Uses}

\begin{tabular}{lll}
\textcolor{red}{Pygame} & \\
\textcolor{red}{Time} & \\
\textcolor{red}{sys} & \\
\textcolor{red}{MenuController} & \\
    Character & \\
    Obstacle & \\
    Powerups & \\
    DetectCollision & \\
    UpdateEnvironment & \\
    Score & \\
    DisplayObstacle & \\
    DisplayPowerups & \\
    DisplayEnvironment & \\
    DisplayWindow & \\
    DisplayCharacter & \\
    PlaySound & \\
    DisplayMenu & \\
    LoadAssets 
\end{tabular}

\subsection* {Syntax}

\subsubsection* {Exported Access Programs}

\begin{tabular}{| l | p{6cm} | l | l |}
\hline
\textbf{Routine name} & \textbf{In} & \textbf{Out} & \textbf{Exceptions}\\
\hline
    GameController & --- & GameController & ---\\
\hline  
    check\_user\_input & --- &\textcolor{red}{String} & \textcolor{red}{---}\\
\hline
    game\_loop & \textcolor{red}{clock(), $\mathbb{R}$} & --- & ---\\
\hline
    \textcolor{red}{increase\_game\_speed} & \textcolor{red}{DisplayPowerups, DisplayObstacles} &\textcolor{red}{---} & \textcolor{red}{---}\\
\hline
    \textcolor{red}{main\_menu} & \textcolor{red}{pygame.font} &\textcolor{red}{String} & \textcolor{red}{---}\\
\hline
    \textcolor{red}{run\_game} & \textcolor{red}{---} &\textcolor{red}{---} & \textcolor{red}{---}\\
\hline
    \textcolor{red}{resume\_game} & \textcolor{red}{
    DisplayObstacles, DisplayEnvironment, DisplayPowerups, DisplayCharacter, seq of $\mathbb{Z}$, $\mathbb{R}$}
     &\textcolor{red}{$\mathbb{R, R}$} & \textcolor{red}{---}\\
\hline
    \textcolor{red}{detect\_powerups\_collision} & \textcolor{red}{DisplauPowerups, DisplayObatcles} &\textcolor{red}{---} & \textcolor{red}{---}\\
\hline
    \textcolor{red}{detect\_obstacles\_collision} & \textcolor{red}{$\mathbb{B, R}$, DisplayObatcles} &\textcolor{red}{$\mathbb{B}$} & \textcolor{red}{---}\\
\hline
\end{tabular}

\subsection* {Semantics}

\subsubsection* {State Variables}

game\_screen: pygame.display\\
\textcolor{red}{obstacle\_img: seq of pygame.image}\\
\textcolor{red}{sound\_list: seq of pygame.mixer}\\
\textcolor{red}{game\_speed: $\mathbb{R}$}\\
\textcolor{red}{load\_character: seq of seq of pygame.image}\\
obstacle\_list: seq of Obstacle\\
character : Character \\
powerup\_list : seq of Powerups\\
\textcolor{red}{play\_sound: PlaySound}\\
\textcolor{red}{menu\_controller: MenuController}\\
\textcolor{red}{pause\_time: $\mathbb{R}$}\\
\sout{background\_music: pygame.mixer}\\
\sout{sound\_effects: pygame.mixer\\}
\textcolor{red}{main\_menu\_img: pygame.image}\\
\textcolor{red}{pause\_menu\_img: pygame.image}\\
\textcolor{red}{end\_menu\_img: pygame.image}\\
\textcolor{red}{setting\_menu\_img: pygame.image}\\
\textcolor{red}{instruction\_menu\_img: pygame.image}\\
floor: pygame.image\\
floor\_position: $\mathbb{N}$\\
background: pygame.image\\
score\_count: Score\\
\textcolor{red}{is\_paused: $\mathbb{B}$}\\

\subsubsection* {Environment Variables}
\textcolor{red}{QUIT: Mouse Input close the window  }\\
\textcolor{red}{K\_DOWN: Keyboard input down arrow}\\
\textcolor{red}{K\_UP: Keyboard input up arrow}\\
\textcolor{red}{K\_p: Keyboard input p button}\\
\textcolor{red}{K\_s: Keyboard input s button}\\
\textcolor{red}{K\_h: Keyboard input h button}\\
\textcolor{red}{K\_SPACE: Keyboard input space bar}\\
\subsubsection* {State Invariant}

\sout{None}\textcolor{red}{$game\_speed \in [INIT\_SPPED, MAX\_SPEED]$}

\subsubsection* {Assumptions}

\sout{None} \textcolor{red}{The constructor shall be called before other methods}

\subsubsection* {Access Routine Semantics}

GameController():
\begin{itemize}
    \item transition: game\_screen := DisplayWindow.get\_game\_screen()\\
    obstacle\_list, \sout{powerup\_list} := \textcolor{red}{seq of all kinds of Obstacles}, \sout{[]}\\
    \textcolor{red}{game\_speed := INIT\_SPEED}\\
    character := Character(game\_screen, LoadAssets.load\_character()[0])\\
    \sout{background, sound\_effects := LoadAssets.load\_sound()\\
    floor := LoadAssets.load\_floor()\\
    background := LoadAsset.load\_background()\\}
    \textcolor{red}{All images, sounds are initialized with corresponding LoadAssets methods}\\
    \textcolor{red}{menu\_controller := MenuController(game\_screen)}\\
    \textcolor{red}{pause\_time := 0}\\
    \textcolor{red}{floor\_position := 0}\\
    \textcolor{red}{is\_puased := False}\\
    score\_count := Score()\\
    \textcolor{red}{play\_sound.play\_bg\_music()}
    \item output: \textit{out} := \textit{self}
\end{itemize}

\noindent check\_user\_input():
\begin{itemize}
    \item transition: Checks for user input and calls the corresponding method. The method is responsible for handling input for character control and moving the character based on the input. The method also handles the user inputs for starting, quitting, and going to the settings from the main menu. 
    \textcolor{red}{\item output: \textit{out} := \textit{User input in pause menu}}\\\\
\resizebox{15cm}{!}{
\begin{tabular}{|l|l|}
    \hline
    Input Key & Behaviour\\
    \hline
    \hline
    pygame.QUIT & System exit\\
    \hline
    \hline
    pygame.KEYDOWN & \textcolor{red}{If any keyboard input is pressed}\\
    \hline\
    pygame.K\_DOWN & character.duck() \& play\_duck\_sound()\\
    pygame.K\_UP & character.jump() \& play\_jump\_sound()\\
    \sout{pygame.K\_Up when jumping} & \sout{character.double\_jump() \&
    play\_jump\_sound()}\\
    pygame.K\_p & MenuController.pause\_menu(game\_screen)\\
    \hline
    \hline
    pygame.KEYUP & \textcolor{red}{If any keyboard input is not pressed}\\
    \hline
    pygame.K\_DOWN & character.stand()\\
    \hline
\end{tabular}
}
\end{itemize}

\noindent game\_loop(\textcolor{red}{clock, game\_start\_time}):
\begin{itemize}
    \item transitions: It is the main game loop of the game that will continuously run until the current game session ends when the user presses the `Quit' button in the main menu. It will call methods from other modules to control the game play. \sout{The floor image will have its position constantly moving and being updated. The events of each keyboard input will be constantly monitored to ensure the user input is registered. The score will be incremented and depending on the current score, the background color will change. There will be random types of obstacles  and powerups generating at random times. There will be a constant check for collision detection between the character and obstacle, and character and powerups. If a collision between a character and obstacle has occured, the current game state stops and goes to the exits menu to compare current score and highest score achieved.} \textcolor{red}{It calls check\_user\_input method to detect input, and gets all objects drawn on the screen and updated every loop. It also detects collision with detectDollision methods and takes corresponding action if collision happens.} 
\end{itemize}

\noindent \textcolor{red}{
increase\_game\_speed(powerups, obstacles)
\begin{itemize}
    \item transition: $game\_speed < MAX\_SPEED \Rightarrow (character.get\_slo\_mo \Rightarrow gamespeed := INIT\_SPEED \land \\ \neg character.get\_slo\_mo \Rightarrow gamespeed := INIT\_SPEED + SPEED\_FACTOR*(score//SCORE\_SPEED)) \\ obsacles.update\_speed(game\_speed); powerups.update\_speed(game\_speed)$
\end{itemize}}

\noindent \textcolor{red}{
main\_menu(font)
\begin{itemize}
    \item transition: It detects all user input in main menu and corresponds to these inputs.
    \item out: \textit{out} := \textit{action}
\end{itemize}
\resizebox{15cm}{!}{
\begin{tabular}{|l|l|}
    \hline
    Input Key & Behaviour\\
    \hline
    \hline
    pygame.KEYDOWN & If any keyboard input is pressed\\
    \hline\
    pygame.K\_SPACE & action = "Play"\\
    pygame.K\_q & stop\_music \& running = False\\
    pygame.K\_s & call menu\_controller setting menu method\\
    pygame.K\_h & call menu\_controller instruction menu method\\
    \hline
\end{tabular}}\\\\}

\noindent \textcolor{red}{
run\_game()
\begin{itemize}
    \item transition: start the game if action is "Play", initialize font, clock and Boolean value running.
\end{itemize}}

\noindent \textcolor{red}{
resume\_game(d\_obstacles, d\_environment, d\_powerups, d\_character, bg\_rgb, game\_start\_time)
\begin{itemize}
    \item transition:  Resume and redrawn all elments in game, show a RESUME\_TIME count down before the game is resumed. 
    \item output: \textit{out} := \textit{game\_start\_time, obstacle\_spawn\_time}
\end{itemize}}

\noindent \textcolor{red}{
detect\_powerups\_collision(d\_powerups, d\_obstacles)
\begin{itemize}
    \item transition: Detect collision between the character and powerups. If there is a collision, play sound, change the status of the character/score) and remove that powerup.
\end{itemize}}

\noindent \textcolor{red}{
detect\_obstacles\_collision(running, current\_score, d\_obstacles)
\begin{itemize}
    \item transition: Detect the collision between obstacles and character and play sound for collisions. If there is a successful collision(with no invincibility powerup),set running to False
    \item out: \textit{out} := \textit{running}
\end{itemize}}

\subsection* {Local Constants}
\textcolor{red}{
    INIT\_SPEED =  $\mathbb{N}$ \\
    MAX\_SPEED =  $\mathbb{N}$ \\
    OBS\_START\_X =  $\mathbb{N}$ \\
    OBS\_START\_Y = $\mathbb{N}$ \\
    INIT\_BG = $[\mathbb{N,N,N}]$ \\
    FONTSIZE = $\mathbb{N}$ \\
    FPS = $\mathbb{N}$ \\
    POWERUPS\_TIME = $\mathbb{N}$ \\
    RESUME\_TIME = $\mathbb{N}$ \\
    SPEED\_FACTOR =  $\mathbb{R}$ \\
    SCORE\_SPEED = $\mathbb{N}$ \\
} 
%%%%%%%%%%%%%%%%%%%%%%%%%%%%%%%%%%%%%%%%%%%%%%%%%%%%%%%%%
\newpage
%%%%%%%%%%%%%%%%%%%%%%%%%%%%%%%%%%%%%%%%%%%%%%%%%%%%%%%%%
\section*{Controller/MenuController}

\subsection* {Uses}

\sout{MainMenu} \textcolor{red}{DisplayMenu}\\
\textcolor{red}{Pygame}\\

\subsection* {Syntax}

\subsubsection* {Exported Access Programs}
\resizebox{17cm}{!}{
\begin{tabular}{| l | l | l | l |}
\hline
\textbf{Routine name} & \textbf{In} & \textbf{Out} & \textbf{Exceptions}\\
\hline
    \textcolor{red}{MenuController} & \textcolor{red}{pygame.display} & \textcolor{red}{MenuController} & \textcolor{red}{IllegalArgumentException}\\
\hline
    setting\_menu & \sout{pygame.display} \textcolor{red}{PlaySound, pygame.image} & --- & ---\\
\hline
    pause\_menu & \sout{pygame.display} \textcolor{red}{pygame.image} & --- & ---\\
\hline
    \sout{exit\_menu} \textcolor{red}{end\_menu} & \sout{pygame.display} \textcolor{red}{$\mathbb{N,N}$, pygame.image} & --- & ---\\
\hline
    \textcolor{red}{resume\_menu} & \textcolor{red}{$\mathbb{N}$} & \textcolor{red}{---}& \textcolor{red}{---}\\
\hline
    \textcolor{red}{instruction\_menu} & \textcolor{red}{pygame.image} & \textcolor{red}{---}& \textcolor{red}{---}\\
\hline
\end{tabular}
}

\subsection* {Semantics}

\subsubsection* {State Variables}
\textcolor{red}{game\_screen: pygame.display}

\subsubsection* {Environment Variables}
\textcolor{red}{QUIT: Mouse Input close the window  }\\
\textcolor{red}{K\_b: Keyboard input b arrow}\\
\textcolor{red}{K\_r: Keyboard input r button}\\
\textcolor{red}{K\_c: Keyboard input c button}\\
\textcolor{red}{K\_q: Keyboard input q button}\\

\subsubsection* {State Invariant}

None

\subsubsection* {Assumptions}

None

\subsubsection* {Access Routine Semantics}
\textcolor{red}{MenuController(window):}
\begin{itemize}
    \item \textcolor{red}{output: $out$ := $self$}
    \item \textcolor{red}{transition: game\_screen := window}
    \item \textcolor{red}{exception: $exc$ := $(window \equiv NULL) \Rightarrow IllegalArgumentException $}
\end{itemize}
\noindent setting\_menu(\sout{screen} \textcolor{red}{play\_sound, setting\_menu\_img}):
\begin{itemize}
    \item transition: Display the settings menu onto the screen and handle the volume \sout{and theme} change according to user input. \textcolor{red}{ The user will be able to modify the sound effect and background volume separately and handle saving these options.} When the user press the `Back' button, the method terminates and control is shifted back to the game controller.
\end{itemize}

\noindent pause\_menu(\sout{screen}, \textcolor{red}{pause\_menu\_img}):
\begin{itemize}
    \item transition: Display the pause menu onto the screen and freeze the current game state. The score count is kept unchanged and all the elements of the game (obstacles, character, and background) stop moving. If the `Resume' button is pressed, the game data current state is unfrozen, the method terminates and the control is shifted back to the game controller. If the Exit' button is pressed, the method terminates and the control is shifted back to the game controller (without saving the current game state upon exiting).
\end{itemize}

\noindent \sout{exit\_menu(screen)} \textcolor{red}{end\_menu(current\_score, highest\_score, end\_menu\_img)}:
\begin{itemize}
    \item transition: Display the exit menu onto the screen and restart or quit the game according to user input. \textcolor{red}{As well display the current\_score and highest\_score onto the screen}. When the user presses the `Exit' button, the game \sout{goes back to main menu} \textcolor{red}{terminates}. If the `Restart' button is pressed, \sout{a new game starts} the method terminates and control is passed back to the game controller \textcolor{red}{to return back to main menu} \sout{whatever button is pressed}.
\end{itemize}

\noindent  \textcolor{red}{resume\_menu(time\_remaining)}:
\begin{itemize}
    \item \textcolor{red}{transition: Display the time remaining before the game resumes back to the current game state where all character, powerups, obstacles and other environment visuals are still visible.}
\end{itemize}

\noindent  \textcolor{red}{instruction\_menu(instruction\_menu\_img)}:
\begin{itemize}
    \item \textcolor{red}{transition: Display the instruction menu onto the screen and handle the input to return back to the main menu. If the `Back' button is pressed, the method terminates and the control is shifted back to the game controller.}
\end{itemize}

\end {document}
