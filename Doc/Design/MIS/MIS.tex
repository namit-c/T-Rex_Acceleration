\documentclass[12pt]{article}

\usepackage{graphicx}
\usepackage{amsmath}
\usepackage{hhline}
\usepackage{booktabs}
\usepackage{multirow}
\usepackage{multicol}
\usepackage{tabularx}
\usepackage[normalem]{ulem}

\usepackage[left=0.70in, right=0.70in, top=0.70in, bottom=0.70in,
  headsep=0pt,% remove space between header and text body
  headheight=17pt% suggested by fancyhdr
  ]{geometry}

\usepackage{paralist}
\usepackage{amsfonts}


\usepackage{url}
\usepackage{longtable}
\usepackage{float}
\usepackage{soul}
\usepackage{xcolor}
\usepackage{ulem}

\oddsidemargin 0mm
\evensidemargin 0mm
\textwidth 160mm
\textheight 200mm
\renewcommand\baselinestretch{1.0}

\pagestyle {plain}
\pagenumbering{arabic}

\newcounter{stepnum}

\usepackage{color}

\newif\ifcomments\commentstrue

\ifcomments
\newcommand{\authornote}[3]{\textcolor{#1}{[#3 ---#2]}}
\newcommand{\todo}[1]{\textcolor{red}{[TODO: #1]}}
\else
\newcommand{\authornote}[3]{}
\newcommand{\todo}[1]{}
\fi

\newcommand{\wss}[1]{\authornote{blue}{SS}{#1}}

\title{SE 3XA3: Module Interface Specification T-Rex Acceleration}

\author{Team 15, Dev\textsuperscript{enthusiasts}
		\\ Zihao Du (duz12)
		\\ Andrew Balmakund (balmakua) 
		\\ Namit Chopra (choprn9)
}

\date{\today}

\begin {document}
 
\maketitle

\begin{table}[bp]
\caption{\bf Revision History}
\begin{tabularx}{\textwidth}{p{3cm}p{2cm}X}
\toprule {\bf Date} & {\bf Version} & {\bf Notes}\\
\midrule
03/18/2021 & 1.0 & Initial Draft\\
\bottomrule
\end{tabularx}
\end{table}

\newpage

\section* {Model/Character}

\subsection* {Uses}
Pygame\\
Time

\subsection* {Syntax}
\subsubsection* {Exported Access Programs}

\resizebox{18cm}{!}{
\begin{tabular}{| l | l | l | l |}
\hline
\textbf{Routine name} & \textbf{In} & \textbf{Out} & \textbf{Exceptions}\\
\hline
Character & pygame.display, pygame.image & Character & IllegalArgumentException \\
\hline
set\_img & pygame.image & --- & IllegalArgumentException\\
\hline
duck & pygame.image, pygame.image & --- & IllegalArgumentException \\
\hline
stand & pygame.image, pygame.image & --- & IllegalArgumentException\\
\hline
jump & pygame.image, pygame.image & --- & IllegalArgumentException\\
\hline
invincible & pygame.image & --- & IllegalArgumentException \\
\hline
double\_jumping & --- & --- & ---\\
\hline
slo\_mo & --- & --- & ---\\
\hline
update & --- & --- & ---\\
\hline

\end{tabular}
}

\subsection* {Semantics}

\subsubsection* {State Variables}

\begin{tabular}{lll}
    game\_screen: pygame.display & // & Represents the screen \\
    img: pygame.image & // & Surface object with the image of the character\\
    is\_ducking: $\mathbb{B}$ & // & Is the character ducking\\
    is\_jumping: $\mathbb{B}$ & // & Is the character jumping\\
    is\_invincible: $\mathbb{B}$ & // & Is the character invincible\\
    is\_slo\_mo: $\mathbb{B}$ & // & Is the character slowing obstacles\\
    is\_double\_jumping: $\mathbb{B}$ & // & Can the character double jump\\
    movement: ($\mathbb{R,R}$) & // & horizontal and vertical speed of the character\\
\end{tabular}

\subsubsection* {Environment Variables}
None
\subsubsection* {State Invariant}

$is\_ducking \land is\_jumping = False$\\
$is\_invincible \land is\_slo\_mo \land is\_double\_jumping = False$\\
\subsubsection* {Assumptions}

Constructor character is called first before other methods.

\subsubsection* {Access Routine Semantics}

\noindent Character(screen, char\_img):
\begin{itemize}
\item transition: game\_screen, img, movement := screen, char\_img, [0,0]\\ All Boolean state variables are initialized to False
\item output: \textit{out} := \textit{self}
\item exception: 
    \begin{itemize}[]
        \item $exc$ := $(img \equiv NULL) \Rightarrow IllegalArguementException$
        \item $exc$ := $(game\_screen \equiv NULL) \Rightarrow IllegalArguementError$
    \end{itemize}
\end{itemize}

\noindent set\_img(new\_img):
\begin{itemize}
    \item transition: img := new\_img
\item exception: 
    \begin{itemize}[]
        \item $exc$ := $(new\_img \equiv NULL) \Rightarrow IllegalArguementException$
    \end{itemize}    
\end{itemize}

\noindent duck(ducking\_img, inv\_ducking\_img):
\begin{itemize}
\item transition: $\neg is\_jumping \Rightarrow (is\_ducking := False;\\
is\_invincible \Rightarrow img := inv\_ducking\_img \land \neg is\_invincible \Rightarrow img := ducking\_img)$
\item exception:
    \begin{itemize}[]
        \item   $exc$ := $(any img \equiv NULL) \Rightarrow IllegalArguementException$
    \end{itemize}
\end{itemize}

\noindent stand(inv\_char, char\_img):
\begin{itemize}
\item transition: $\neg is\_jumping \Rightarrow (is\_ducking := False;\\
is\_invincible \Rightarrow img := inv\_char \land \neg is\_invincible \Rightarrow img := char\_img)$
\item exception:
    \begin{itemize}[]
        \item   $exc$ := $(any img \equiv NULL) \Rightarrow IllegalArguementException$
    \end{itemize}
\end{itemize}

\noindent jump(inv\_jumping\_char, jumping\_img):
\begin{itemize}
\item transition: $\neg (is\_ducking \land is\_jumping) \Rightarrow (is\_jumping, movement[1] := True, INIT\_SPEED)$\\
\item exception: $exc$ := $(any img \equiv NULL) \Rightarrow IllegalArguementException$
\end{itemize}

\noindent invincible(inv\_char):
\begin{itemize}
\item transition: is\_invincible, img, is\_double\_jumping, is\_slo\_mo := True, inv\_char, False, False\\
\item exception:
    \begin{itemize}[]
        \item   $exc$ := $(inv\_char \equiv NULL) \Rightarrow IllegalArguementException$
    \end{itemize}
\end{itemize}

\noindent double\_jump():
\begin{itemize}
\item transition: is\_double\_jumping, is\_slo\_mo, is\_invincible := True, False, False\\
\end{itemize}

\noindent slo\_mo():
\begin{itemize}
\item transition: is\_slo\_mo, is\_invincible, is\_double\_jumping := True, False, False\\
\end{itemize}

\noindent update():
\begin{itemize}
\item transition: Change is\_double\_jumping or is\_invincible or is\_slo\_mo to False and the corresponding image back if the power up has already lasted for DURATION\_TIME. If is\_jumping is True, decrease movement[1] by GRAVITY, and check if the jumping ends(if the character back to ground)
\end{itemize}
\subsection*{Local Constants}
INIT\_SPEED = -20\\
GRAVITY = 1\\
DURATION\_TIME = 5\\
\medskip
%%%%%%%%%%%%%%%%%%%%%%%%%%%%%%%%%%%%%%%%%%%%%%%%%%%%%%%%%%%%%%%%%%
\newpage
%%%%%%%%%%%%%%%%%%%%%%%%%%%%%%%%%%%%%%%%%%%%%%%%%%%%%%%%%%%%%%%%%%

\section*{Model/Powerups}

\subsection* {Uses}

Pygame

\subsection* {Syntax}

\subsubsection* {Exported Access Programs}

\begin{tabular}{| l | l | l | l |}
\hline
\textbf{Routine name} & \textbf{In} & \textbf{Out} & \textbf{Exceptions}\\
\hline
    Powerups & $\mathbb{N, N, R}$, pygame.image & Powerups & IllegalArguementException\\
\hline
    get\_width & --- & $\mathbb{N}$ & ---\\
\hline
    set\_width & $\mathbb{N}$ & --- & IllegalArguementException\\
\hline
    get\_height & --- & $\mathbb{N}$ & ---\\
\hline
    set\_height & $\mathbb{N}$ & --- & IllegalArguementException\\
\hline
    get\_speed & --- & $\mathbb{R}$ & --- \\
\hline
    set\_speed & $\mathbb{N}$ & ---& --- \\
\hline
    get\_img & --- & string & ---\\
\hline
    set\_img & pygame.img & --- & IllegalArguementException\\
\hline
    get\_name & --- & String & --- \\
\hline
\end{tabular}

\subsection* {Semantics}

\subsubsection* {State Variables}
\begin{tabular}{lll}
name: String & // & Represents the type of a power up\\
    width: $\mathbb{N}$ & // & Represents the width of a power up\\
    height: $\mathbb{N}$ & // & Represents the height of a power up\\
    speed: $\mathbb{R}$ & // & Represents the speed of a power up\\
    img: pygame.image & // & Surface object with a specified image drawn onto it\\
\end{tabular}

\subsubsection* {State Invariant}

None

\subsubsection* {Assumptions \& Design Decisions}

None

\subsubsection* {Access Routine Semantics}

\noindent Powerups(width, height, speed, powerup\_img):
\begin{itemize}
    \item transition: width, height, speed, img, name := width, height, speed, powerup\_img, random element in POWERUP\_LIST
    \item output: $out := self$
   \item exception: $exc$ := $(powerup\_img \equiv NULL) \Rightarrow IllegalArguementException$
\end{itemize}

\noindent get\_width():
\begin{itemize}
    \item output: $out$ := width 
\end{itemize}


\noindent set\_width(new\_width):
\begin{itemize}
    \item transition: width := new\_width 
    \item exception: $exc$ := ($\text{new\_width} < 0$) $\Rightarrow  IllegalArgumentException$
\end{itemize}

\noindent get\_height():
\begin{itemize}
    \item output: $out$ := height
\end{itemize}


\noindent set\_height(new\_height):
\begin{itemize}
    \item transition: height := new\_height
    \item exception: $exc$ := ($\text{new\_height} < 0$) $\Rightarrow  IllegalArgumentException$
\end{itemize}


\noindent get\_speed():
\begin{itemize}
    \item output:$out$ := speed
\end{itemize}


\noindent set\_speed(new\_speed):
\begin{itemize}
    \item transition:speed := new\_speed
\end{itemize}


\noindent get\_img():
\begin{itemize}
    \item output: $out$ := img 
\end{itemize}


\noindent set\_img(new\_powerup\_img):
\begin{itemize}
    \item transition: img := new\_powerup\_img
    \item exception: $exc$ := $(new\_powerup\_img \equiv NULL) \Rightarrow IllegalArgumentException$
\end{itemize}

\noindent get\_name():
\begin{itemize}
    \item output: $out$ := name
\end{itemize}

\subsection*{Local Constants}
POWERUP\_LIST = [``Invincibility", ``Double Jump", ``Score Boost", ``Slo\_mo"]
%%%%%%%%%%%%%%%%%%%%%%%%%%%%%%%%%%%%%%%%%%%%%%%%%%%%%%%%%
\newpage
%%%%%%%%%%%%%%%%%%%%%%%%%%%%%%%%%%%%%%%%%%%%%%%%%%%%%%%%%

\section*{Model/Obstacle}

\subsection* {Uses}
Pygame


\subsection* {Syntax}

\subsubsection* {Exported Access Programs}

\begin{tabular}{| l | l | l | l |}
\hline
\textbf{Routine name} & \textbf{In} & \textbf{Out} & \textbf{Exceptions}\\
\hline
    Obstacle & String,$\mathbb{N, N, R}$, pygame.image & Obstacle & ---\\
\hline
    get\_width & --- & $\mathbb{N}$ & ---\\
\hline
    set\_width & $\mathbb{N}$ & --- & IllegalArgumentException\\
\hline
    get\_height & --- & $\mathbb{N}$ & ---\\
\hline
    set\_height & $\mathbb{N}$ & --- & IllegalArgumentException\\
\hline
    get\_speed & --- & $\mathbb{R}$ & --- \\
\hline
    set\_speed & $\mathbb{R}$ & ---& --- \\
\hline
    get\_img & --- & pygame.image & ---\\
\hline
    set\_img & pygame.image & --- & IllegalArgumentException\\

\hline
    get\_rect & --- & ($\mathbb{R,R}$) & \\
\hline
    set\_rect & ($\mathbb{R,R}$) & --- & ---\\
\hline
\end{tabular}

\subsection* {Semantics}

\subsubsection* {State Variables}
\begin{tabular}{lll}
name: String & // & Represents the name of an obstacle\\
    width: $\mathbb{N}$ & // & Represents the width of an obstacle\\
    height: $\mathbb{N}$ & // & Represents the height of an obstacle\\
    speed: $\mathbb{R}$ & // & Represents the speed of an obstacle\\
    img: pygame.image & // & Surface object with a specified image drawn onto it\\
    rect: ($\mathbb{R,R}$) & // & X and Y coordinates of the obstacle\\
\end{tabular}

\subsubsection* {State Invariant}

None

\subsubsection* {Assumptions \& Design Decisions}

None

\subsubsection* {Access Routine Semantics}

\noindent Obstacle(name, width, height, speed, obstacle\_img):
\begin{itemize}
    \item transition:  name, width, height, speed, img, rect:= name, width, height, speed, obstacle\_img, obstacle\_img.get\_rect()
    \item output: $out := self$
\end{itemize}

\noindent get\_width():
\begin{itemize}
    \item output: $out$ := width 
\end{itemize}


\noindent set\_width(new\_width):
\begin{itemize}
    \item transition: width := new\_width 
    \item exception: $exc$ := ($\text{new\_width} < 0$) $\Rightarrow  IllegalArgumentException$
\end{itemize}

\noindent get\_height():
\begin{itemize}
    \item output: $out$ := height
\end{itemize}


\noindent set\_height(new\_height):
\begin{itemize}
    \item transition: height := new\_height
    \item exception: $exc$ := ($\text{new\_height} < 0$) $\Rightarrow  IllegalArgumentException$
\end{itemize}


\noindent get\_speed():
\begin{itemize}
    \item output: $out$ := speed
\end{itemize}


\noindent set\_speed(new\_speed):
\begin{itemize}
    \item transition: speed := new\_speed
\end{itemize}


\noindent get\_img():
\begin{itemize}
    \item output: $out$ := img 
\end{itemize}


\noindent set\_img(new\_osbtacle\_img):
\begin{itemize}
    \item transition: img := new\_osbtacle\_img
    \item exception: $exc$ := $(new\_osbtacle\_img \equiv NULL) \Rightarrow IllegalArgumentException $
\end{itemize}

\noindent get\_rect():
\begin{itemize}
    \item output: $out$ := rect 
\end{itemize}


\noindent set\_rect(x, y):
\begin{itemize}
    \item transition: rect.x, rect.y := x, y
\end{itemize}



%%%%%%%%%%%%%%%%%%%%%%%%%%%%%%%%%%%%%%%%%%%%%%%%%%%%%%%%%
\newpage
%%%%%%%%%%%%%%%%%%%%%%%%%%%%%%%%%%%%%%%%%%%%%%%%%%%%%%%%%
\section*{Model/DetectCollision}

\subsection* {Uses}

Pygame

\subsection* {Syntax}

\subsubsection* {Exported Access Programs}

\begin{tabular}{| l | l | l | l |}
\hline
\textbf{Routine name} & \textbf{In} & \textbf{Out} & \textbf{Exceptions}\\
\hline
    detect\_collision & pygame.sprite, seq of pygame.sprite & $\mathbb{B}$  & ---\\
\hline
    find\_collision & pygame.sprite, seq of pygame.sprite & pygame.sprite  & ---\\
\hline
\end{tabular}

\subsection* {Semantics}

\subsubsection* {State Variables}

None

\subsubsection* {State Invariant}

None

\subsubsection* {Assumptions \& Design Decisions}

All pygame.sprite objects have been defined have been defined.

\subsubsection* {Access Routine Semantics}

detect\_collision(character, elements):
\begin{itemize}
    \item output: $out$ := ($\forall \text{ element} \in \text{elements } | \text{ (character.X}< \text{element.X} + \text{element.width}) \land (\text{character.X} + \text{character.width} > \text{element.X}) \land (\text{character.Y} < \text{element.Y} + \text{element.height}) \land (\text{character.Y} + \text{Character.height} > \text{element.Y})$)
\end{itemize}

\noindent find\_collision(character, elements):
\begin{itemize}
    \item output: $out$ := the element in the sequence elements that collides with the character.
\end{itemize}


%%%%%%%%%%%%%%%%%%%%%%%%%%%%%%%%%%%%%%%%%%%%%%%%%%%%%%%%%
\newpage
%%%%%%%%%%%%%%%%%%%%%%%%%%%%%%%%%%%%%%%%%%%%%%%%%%%%%%%%%
\section*{Model/UpdateEnvironment}

\subsection* {Uses}
Pygame

\subsection* {Syntax}

\subsubsection* {Exported Access Programs}

\begin{tabular}{| l | l | l | l |}
\hline
\textbf{Routine name} & \textbf{In} & \textbf{Out} & \textbf{Exceptions}\\
\hline
    update\_floor & $\mathbb{Z}$ & $\mathbb{Z}$ & ---\\
\hline
    update\_bg\_colour & $\mathbb{Z}$, $\mathbb{Z}$, seq of $\mathbb{Z}$ & seq of $\mathbb{Z}$ & ---\\
\hline
\end{tabular}

\subsection* {Semantics}

\subsubsection* {State Variables}

None

\subsubsection* {State Invariant}

None

\subsubsection* {Assumptions}

The game window and images are defined before any routines are called.

\subsubsection* {Access Routine Semantics}

update\_floor(floor\_position):
\begin{itemize}
    \item output: \textit{out} := (floor\_position $\leq$ -500) $\Rightarrow$ floor\_position = 0 $\lor$ (floor\_position = floor\_position - MOVEMENT\_SPEED)
\end{itemize}
\noindent update\_bg\_colour(current\_score, previous\_score, bg\_rgb):
\begin{itemize}
    \item output: \textit{out} := Every time the score is a multiple of CHANGE\_BG, the value of either red (bg\_rbg[0]), green (bg\_rbg[1]), or blue (bg\_rbg[2]) changes. The new value of one of the field is: $(0 \leq x \leq 2 |$ $bg\_rgb[x] := (bg\_rgb[x]+50)\%255 )$
\end{itemize}
\subsubsection* {Local Constants}
MOVEMENT\_SPEED: 10 \\
CHANGE\_BG = 50
%%%%%%%%%%%%%%%%%%%%%%%%%%%%%%%%%%%%%%%%%%%%%%%%%%%%%%%%%
\newpage
%%%%%%%%%%%%%%%%%%%%%%%%%%%%%%%%%%%%%%%%%%%%%%%%%%%%%%%%%
\section*{Model/Score}

\subsection* {Uses}

Time

\subsection* {Syntax}

\subsubsection* {Exported Access Programs}

\begin{tabular}{| l | l | l | l |}
\hline
\textbf{Routine name} & \textbf{In} & \textbf{Out} & \textbf{Exceptions}\\
\hline
    Score &--- & Score& ---\\
\hline 
    get\_score & --- & $\mathbb{N}$ & --- \\
\hline
    update\_score & --- & --- & ---\\
\hline
    get\_high\_score & --- & $\mathbb{N}$ & --- \\
\hline
\end{tabular}

\subsection* {Semantics}

\subsubsection* {State Variables}

high\_score: $\mathbb{N}$ \\
current\_score: $\mathbb{N}$ \\
previous\_score: $\mathbb{N}$ \\
start\_time: $\mathbb{R}$

\subsubsection* {State Invariant}

None

\subsubsection* {Assumptions}

None

\subsubsection* {Access Routine Semantics}

Score():
\begin{itemize}
    \item output: $out$ := \textit{self} 
    \item transition: high\_score, current\_score, start\_time := 0, 0, current time
\end{itemize}
\noindent get\_score():
\begin{itemize}
    \item output: $out$ := current\_score
\end{itemize}
\noindent update\_score():
\begin{itemize}
    \item output: $out$ := current\_score, previous\_score
    \item transition: previous\_score, current\_score, high\_score := current\_score, ((current time - start\_time)$\cdot$ SCALE\_FACTOR) rounded to the nearest natural number, (current\_score $>$ high\_score) $\Rightarrow$ current\_score $\lor$ high\_score
\end{itemize}
\noindent get\_score():
\begin{itemize}
    \item output: $out$ := high\_score
\end{itemize}
\subsubsection* {Local Constants}
SCALE\_FACTOR = 5

%%%%%%%%%%%%%%%%%%%%%%%%%%%%%%%%%%%%%%%%%%%%%%%%%%%%%%%%%
\newpage
%%%%%%%%%%%%%%%%%%%%%%%%%%%%%%%%%%%%%%%%%%%%%%%%%%%%%%%%%
\section*{Model/MainMenu}

\subsection* {Uses}
None

\subsection* {Syntax}

\subsubsection* {Exported Access Programs}

\begin{tabular}{| l | l | l | l |}
\hline
\textbf{Routine name} & \textbf{In} & \textbf{Out} & \textbf{Exceptions}\\
\hline
    MainMenu & --- & MainMenu & ---\\
\hline
    change\_volume & $\mathbb{N}$, $\mathbb{N}$ & --- & ---\\
\hline
    get\_volumes & --- & $\mathbb{N}$, $\mathbb{N}$ & ---\\
\hline
\end{tabular}

\subsection* {Semantics}

\subsubsection* {State Variables}

background\_music\_volume: $\mathbb{N}$\\
sound\_effects\_volume: $\mathbb{N}$

\subsubsection* {State Invariant}

None

\subsubsection* {Assumptions}

None

\subsubsection* {Access Routine Semantics}

MainMenu():
\begin{itemize}
    \item output: $out$ := self
    \item transition: background\_music\_volume, sound\_effects\_volume := MAX\_VOLUME, \\ MAX\_VOLUME
\end{itemize}
\noindent change\_volume(new\_background\_volume, new\_sound\_effects\_volume):
\begin{itemize}
    \item transition: background\_music\_volume, sound\_effects\_volume := new\_background\_volume, new\_sound\_effects\_volume
\end{itemize}
\noindent get\_volumes():
\begin{itemize}
    \item output: $out$ := background\_music\_volume, sound\_effects\_volume 
\end{itemize}
\subsection* {Local Constants}
MAX\_VOLUME: 100

%%%%%%%%%%%%%%%%%%%%%%%%%%%%%%%%%%%%%%%%%%%%%%%%%%%%%%%%%
\newpage
%%%%%%%%%%%%%%%%%%%%%%%%%%%%%%%%%%%%%%%%%%%%%%%%%%%%%%%%%

\section*{View/DisplayObstacle}

\subsection* {Uses}

Pygame\\
Time\\
Obstacle\\
Random

\subsection* {Syntax}

\subsubsection* {Exported Access Programs}

\resizebox{17cm}{!}{
\begin{tabular}{| l | l | l | l |}
\hline
\textbf{Routine name} & \textbf{In} & \textbf{Out} & \textbf{Exceptions}\\
\hline
    DisplayObstacle & pygame.display & DisplayObstacle& IllegalArgumentException \\
\hline
    get\_obstacle\_list & --- & seq of Obstacle  & ---\\
\hline
    remove\_obstacle & Obstacle &  --- & ---\\
\hline
    generate\_obstacle &  seq of Obstacle, $\mathbb{R}$ &  $\mathbb{R}$ & ---\\
\hline
    draw\_obstacle & $\mathbb{R,R}$, Obstacle &  --- & ---\\
\hline
    update\_obstacle\_display & --- &  --- & ---\\
\hline

\end{tabular}
}

\subsection* {Semantics}

\subsubsection* {State Variables}

game\_screen: pygame.display \\
obtacle\_list: seq of Obstacle

\subsubsection* {State Invariant}

None

\subsubsection* {Assumptions}

None

\subsubsection* {Access Routine Semantics}

   DisplayObstacle(window):
\begin{itemize}
    \item output: \textit{out} := \textit{self}
    \item transition: game\_screen, obstacle\_list := window, []
    \item exception: \textit{exc}: (game\_screen = NULL $\Rightarrow$ IllegalArgumentExcpetion)
\end{itemize}
\noindent get\_obstacle\_list():
\begin{itemize}
    \item output: \textit{out} := obstacle\_list
\end{itemize}
\noindent remove\_obstacle(obstacle): 
\begin{itemize}
    \item transition: obstacle\_list := obstacle\_list - obstacle.
\end{itemize}
\noindent draw\_obstacle(current\_x, current\_y, obstacle): 
\begin{itemize}
    \item transition: game\_screen := Draw the given obstacle at a specific location (X, Y) coordinates using the obstacle image.
\end{itemize}
\noindent generate\_obstacle(type\_of\_obstacles, prev\_obstacle\_spawn\_time):
\begin{itemize}
    \item transition: Generate a random kind of obstacle after APPROPRIATE\_TIME and add it to the obstacle\_list. 
    \item out: \textit{out} := current time
\end{itemize}
\noindent update\_obstacle\_display(): 
\begin{itemize}
    \item transition: For each obstacle in obstacle\_list, draw each obstacle at its new position by considering the speed of each object. Once the obstacle is outside the boundaries of the game\_screen, remove the obstacle from the list.
\end{itemize}

\subsubsection* {Local Constants}
APPROPRIATE\_TIME = random integer between 3 and 5 seconds
%%%%%%%%%%%%%%%%%%%%%%%%%%%%%%%%%%%%%%%%%%%%%%%%%%%%%%
\newpage 
%%%%%%%%%%%%%%%%%%%%%%%%%%%%%%%%%%%%%%%%%%%%%%%%%%%%%%%%%%%%%%%
\section*{View/DisplayPowerups}

\subsection* {Uses}

Pygame\\
Powerups\\
Random

\subsection* {Syntax}

\subsubsection* {Exported Access Programs}

\resizebox{15cm}{!}{
\begin{tabular}{| l | l | l | l |}
\hline
\textbf{Routine name} & \textbf{In} & \textbf{Out} & \textbf{Exceptions}\\
\hline
    DisplayPowerups & pygame.display & DisplayPowerups & IllegalArgumentException \\
\hline
    generate\_powerup & --- & --- & ---\\
\hline
    draw\_powerups & Powerups &  --- & ---\\
\hline
    update\_powerup & --- & --- & ---\\
\hline

\end{tabular}
}

\subsection* {Semantics}

\subsubsection* {State Variables}

powerups\_diplayed: seq of Powerups\\ 
game\_screen: pygame.display \\

\subsubsection* {State Invariant}

None

\subsubsection* {Assumptions}

None

\subsubsection* {Access Routine Semantics}

display\_powerup(game\_screen):
\begin{itemize}
    \item output: \textit{out} := \textit{self}
    \item transition: game\_screen, powerups\_diplayed := game\_screen, []
    \item exception: \textit{exc}: (game\_screen = NULL $\Rightarrow$ IllegalArgumentExcpetion)
\end{itemize}
\noindent generate\_powerup():
\begin{itemize}
    \item transition: powerups\_diplayed := Add a random kind of powerup in the powerups\_displayed (a list of powerups displayed on the game\_screen). 
\end{itemize}
\noindent draw\_powerups(powerup): 
\begin{itemize}
    \item transition: draw the power up on the screen
\end{itemize}

\noindent update\_powerup():
\begin{itemize}
    \item transition: all elements in powerups\_displayed are updated by position and drawn. 
\end{itemize}


%%%%%%%%%%%%%%%%%%%%%%%%%%%%%%%%%%%%%%%%%%%%%%%%%%%%%%%%%
\newpage
%%%%%%%%%%%%%%%%%%%%%%%%%%%%%%%%%%%%%%%%%%%%%%%%%%%%%%%%%
\section*{View/DisplayEnvironment}

\subsection* {Uses}
UpdateEnvironment \\
Pygame

\subsection* {Syntax}

\subsubsection* {Exported Access Programs}

\begin{tabular}{| l | l | l | l |}
\hline
\textbf{Routine name} & \textbf{In} & \textbf{Out} & \textbf{Exceptions}\\
\hline
    DisplayEnvironment & pygame.display & DisplayEnvironment & IllegalArgumentException\\
\hline
    draw\_score & Score & --- & ---\\
\hline
    display\_instruction & String & --- & ---\\
\hline
    draw\_floor & pygame.image $\mathbb{Z}$ & --- & IllegalArgumentException\\
\hline
    draw\_background & pygame.image & --- & IllegalArgumentException\\
\hline
\end{tabular}

\subsection* {Semantics}

\subsubsection* {State Variables}

None

\subsubsection* {State Invariant}

game\_screen: pygame.display

\subsubsection* {Assumptions}

None

\subsubsection* {Access Routine Semantics}
\noindent DisplayEnvironment(window):
\begin{itemize}
    \item output: $out := self$
    \item transition: game\_screen := window
    \item exception: $exc$ := $(window \equiv NULL) \Rightarrow IllegalArgumentException $
\end{itemize}

\noindent draw\_score(score):
\begin{itemize}
    \item transition: The score is drawn on the game\_screen.
\end{itemize}

\noindent draw\_instruction(instructions):
\begin{itemize}
    \item transition: The instruction is shown on the screen for TIME and disappears after.
\end{itemize}

\noindent draw\_floor(floor, floor\_position):
\begin{itemize}
    \item transition : Draw the floor onto the game\_screen.
    \item exception: $exc$ := $(floor \equiv NULL) \Rightarrow IllegalArgumentException $
\end{itemize}

\noindent draw\_background(background):
\begin{itemize}
    \item transition : Draw the background img onto the game\_screen
    \item exception: $exc$ := $(background \equiv NULL) \Rightarrow IllegalArgumentException $
\end{itemize}

\subsection*{Local Constants}
TIME = 5 seconds\\\
%%%%%%%%%%%%%%%%%%%%%%%%%%%%%%%%%%%%%%%%%%%%%%%%%%%%%%%%%
\newpage
%%%%%%%%%%%%%%%%%%%%%%%%%%%%%%%%%%%%%%%%%%%%%%%%%%%%%%%%%
\section*{View/DisplayWindow}

\subsection* {Uses}
Pygame

\subsection* {Syntax}

\subsubsection* {Exported Access Programs}

\begin{tabular}{| l | l | l | l |}
\hline
\textbf{Routine name} & \textbf{In} & \textbf{Out} & \textbf{Exceptions}\\
\hline
    DisplayWindow & --- & DisplayWindow & ---\\
\hline
    get\_game\_screen &--- & pygame.display & ---\\
% \hline
%     MainMenu & --- & MainMenu & ---\\
\hline
\end{tabular}

\subsection* {Semantics}

\subsubsection* {State Variables}

game\_screen: pygame.display 

\subsubsection* {State Invariant}

None

\subsubsection* {Assumptions}

None

\subsubsection* {Access Routine Semantics}

DisplayWindow():
\begin{itemize}
    \item output: $out$ := $self$
    \item transition: game\_screen := a new pygame window with dimensions WIDTH $\times$ HEIGHT pixels
\end{itemize}
\noindent get\_game\_screen():
\begin{itemize}
    \item output: $out$ := game\_screen
\end{itemize}
\subsection* {Local Constants}
WIDTH = 800 \\
HEIGHT = 600 
%%%%%%%%%%%%%%%%%%%%%%%%%%%%%%%%%%%%%%%%%%%%%%%%%%%%%%%%%
\newpage
%%%%%%%%%%%%%%%%%%%%%%%%%%%%%%%%%%%%%%%%%%%%%%%%%%%%%%%%%
\section*{View/DisplayCharacter}

\subsection* {Uses}
Character
\subsection* {Syntax}

\subsubsection* {Exported Access Programs}

\begin{tabular}{| l | l | l | l |}
\hline
\textbf{Routine name} & \textbf{In} & \textbf{Out} & \textbf{Exceptions}\\
\hline
    DisplayCharacter & pygame.display, Character & --- & ---\\
\hline
    draw\_character & --- & --- & ---\\
\hline
\end{tabular}

\subsection* {Semantics}

\subsubsection* {State Variables}

game\_screen: pygame.display \\
game\_character: Character \\

\subsubsection* {State Invariant}

None

\subsubsection* {Assumptions}

None

\subsubsection* {Access Routine Semantics}
DisplayCharacter(window, character):
\begin{itemize}
    \item transition: game\_screen, game\_character := window, character 
\end{itemize}
\noindent draw\_character():
\begin{itemize}
    \item transition: draw the character onto the game\_screen 
\end{itemize}
%%%%%%%%%%%%%%%%%%%%%%%%%%%%%%%%%%%%%%%%%%%%%%%%%%%%%%%%%
\newpage
%%%%%%%%%%%%%%%%%%%%%%%%%%%%%%%%%%%%%%%%%%%%%%%%%%%%%%%%%
\section*{View/PlaySound}

\subsection* {Uses}
LoadAssets
\subsection* {Syntax}

\subsubsection* {Exported Access Programs}

\begin{tabular}{| l | l | l | l |}
\hline
\textbf{Routine name} & \textbf{In} & \textbf{Out} & \textbf{Exceptions}\\
\hline
    PlaySound & --- & PlaySound & ---\\
\hline
    play\_bg\_music & --- & --- & ---\\
\hline
    play\_jump\_sound & --- & --- & ---\\
\hline
    play\_duck\_sound & --- & --- & ---\\
\hline
    play\_collision\_sound & --- & --- & ---\\
\hline
    play\_powerup\_sound & --- & --- & --- \\
\hline
\end{tabular}

\subsection* {Semantics}

\subsubsection* {State Variables}

background\_music: pygame.mixer\\
jump\_sound: pygame.mixer\\
duck\_sound: pygame.mixer\\
collision\_sound: pygame.mixer\\
powerup\_pickup\_sound: pygame.mixer 

\subsubsection* {State Invariant}

None

\subsubsection* {Assumptions}

None

\subsubsection* {Access Routine Semantics}

PlaySound():
\begin{itemize}
    \item output: $out$ := $self$
    \item transition: background\_music, jump\_sound, duck\_sound, collision\_sound, powerup\_pickup\_sound := the assets are loaded in from the LoadAssets module.
\end{itemize}
\noindent  play\_bg\_music():
\begin{itemize}
    \item transition: plays the background music.
\end{itemize}
\noindent play\_jump\_sound():
\begin{itemize}
    \item transition: plays the jump sound effect.
\end{itemize}
\noindent  play\_duck\_sound():
\begin{itemize}
    \item transition: plays the duck sound effect.
\end{itemize}
\noindent  play\_collision\_sound():
\begin{itemize}
    \item transition: plays the collision sound effect.
\end{itemize}
\noindent  play\_powerup\_sound():
\begin{itemize}
    \item transition: plays the powerup pickup sound effect.
\end{itemize}
%%%%%%%%%%%%%%%%%%%%%%%%%%%%%%%%%%%%%%%%%%%%%%%%%%%%%%%%%
\newpage
%%%%%%%%%%%%%%%%%%%%%%%%%%%%%%%%%%%%%%%%%%%%%%%%%%%%%%%%%
\section*{View/DisplayMenu}

\subsection* {Uses}
MainMenu

\subsection* {Syntax}

\subsubsection* {Exported Access Programs}

\begin{tabular}{| l | l | l | l |}
\hline
\textbf{Routine name} & \textbf{In} & \textbf{Out} & \textbf{Exceptions}\\
\hline
    DisplayMenu & pygame.display & DisplayMenu & IllegalArgumentException\\
\hline
    display\_main\_menu & --- & --- & ---\\
\hline
    display\_pause\_menu & --- & --- & ---\\
\hline
    display\_exit\_menu & --- & --- & ---\\
\hline
     display\_setting\_menu & --- & --- & ---\\
\hline
\end{tabular}

\subsection* {Semantics}

\subsubsection* {State Variables}

game\_screen: pygame.display

\subsubsection* {State Invariant}

None

\subsubsection* {Assumptions}

None

\subsubsection* {Access Routine Semantics}

DisplayMenu(window):
\begin{itemize}
    \item output: $out$ := $self$
    \item transition: game\_screen := window
    \item exception: $exc$ := $(window \equiv NULL) \Rightarrow IllegalArgumentException $
\end{itemize}

\noindent display\_main\_menu():
\begin{itemize}
    \item transition: game\_screen := Main menu with `Play', `Quit' and `Setting' buttons.
\end{itemize}

\noindent display\_pause\_menu():
\begin{itemize}
    \item transition: game\_screen := Pause menu with `Resume' button to resume back to the current game or `Exit' button to go to the main menu.
\end{itemize}

\noindent display\_exit\_menu():
\begin{itemize}
    \item transition: game\_screen := Exit menu with `Return` button to the main menu after game session has ended.
\end{itemize}

\noindent display\_setting\_menu():
\begin{itemize}
    \item transition: game\_screen := Setting menu that can be used to change the volume and theme.
\end{itemize}
%%%%%%%%%%%%%%%%%%%%%%%%%%%%%%%%%%%%%%%%%%%%%%%%%%%%%%%%%
\newpage
%%%%%%%%%%%%%%%%%%%%%%%%%%%%%%%%%%%%%%%%%%%%%%%%%%%%%%%%%
\section*{View/LoadAssets}

\subsection* {Uses}

Pygame


\subsection* {Syntax}

\subsubsection* {Exported Access Programs}

\begin{tabular}{| l | l | l | l |}
\hline
\textbf{Routine name} & \textbf{In} & \textbf{Out} & \textbf{Exceptions}\\
\hline
    load\_floor & --- & pygame.image & IllegalArugementException\\
\hline
    load\_background & --- & pygame.image & IllegalArugementException\\
\hline
    load\_character & --- & seq of pygame.image & IllegalArugementException\\
\hline
load\_all\_obstacle & --- & seq of pygame.image & IllegalArugementException\\
\hline
load\_all\_powerups & --- &  seq of pygame.image & IllegalArugementException\\
\hline
load\_main\_menu & --- &   pygame.image & IllegalArugementException\\
\hline
load\_sound & --- &   pygame.mixer & IllegalArugementException\\
\hline
\end{tabular}

\subsection* {Semantics}

\subsubsection* {State Variables}

None

\subsubsection* {State Invariant}

None

\subsubsection* {Assumptions}

All the files are in the appropriate directory with proper names and format.

\subsubsection* {Access Routine Semantics}

load\_floor():
\begin{itemize}
    \item output: \textit{out} := returns a Pygame image object with the floor image loaded in.
    \item exception: \textit{exc} := $(\text{FLOOR\_IMG} \equiv \neg FileExists) \Rightarrow FileNotFoundError$
\end{itemize}

\noindent load\_background():
\begin{itemize}
    \item output: \textit{out} := returns a Pygame image object with the background image loaded in.
    \item exception: \textit{exc} := $(\text{BACKGROUND\_IMG} \equiv \neg FileExists) \Rightarrow FileNotFoundError$
\end{itemize}

\noindent load\_character():
\begin{itemize}
    \item output: \textit{out} := returns a sequence of Pygame image objects with images of different characters loaded in.
    \item exception: \textit{exc} := $(\text{CHARACTER\_IMG} \equiv \neg FileExists) \Rightarrow FileNotFoundError$
\end{itemize}

\noindent load\_all\_obstacle():
\begin{itemize}
    \item output: \textit{out} := returns a sequence of Pygame image objects with all obstacle images loaded in.
    \item exception: \textit{exc} := $(\text{OBSTACLE\_IMG} \equiv \neg FileExists) \Rightarrow FileNotFoundError$
\end{itemize}

\noindent load\_all\_powerups():
\begin{itemize}
    \item output: \textit{out} := returns a Pygame image object with the all powerup images loaded in.
    \item exception: \textit{exc} := $(\text{POWERUP\_IMG} \equiv \neg FileExists) \Rightarrow FileNotFoundError$
\end{itemize}

\noindent load\_main\_menu():
\begin{itemize}
    \item output: \textit{out} := returns a Pygame image object with the main menu image loaded in.
    \item exception: \textit{exc} := $(\text{MAINMENU\_IMG} \equiv \neg FileExists) \Rightarrow FileNotFoundError$
\end{itemize}

\noindent load\_sound():
\begin{itemize}
    \item output: \textit{out} := returns Pygame sound objects with the background music and sound effects loaded in.
    \item exception: \textit{exc} := $(\text{SOUND\_MP3} \equiv \neg FileExists) \Rightarrow FileNotFoundError$
\end{itemize}
\subsection*{Local Constants}
FLOOR\_IMG = `floor.png' \\
BACKGROUND\_IMG = `background.png' \\
CHARACTER\_IMG = [`character.png', `character\_invisible.png', 'character\_slomo.png']\\
OBSTACLE\_IMG = [`obstacle1.png', `obstacle2.png']\\
POWERUP\_IMG = [`powerup1.png', `powerup2.png', `powerup3.png', `powerup4.png']\\
MAINMENU\_IMG = `mainmenu.png'\\ 
SOUND\_MP3 = [`sound1.mp3', `sound2.mp3', `sound3.mp3', `sound4.mp3', `sound5.mp3']\\

\medskip

%%%%%%%%%%%%%%%%%%%%%%%%%%%%%%%%%%%%%%%%%%%%%%%%%%%%%%%%%
\newpage
%%%%%%%%%%%%%%%%%%%%%%%%%%%%%%%%%%%%%%%%%%%%%%%%%%%%%%%%%
\section*{Controller/GameController}

\subsection* {Uses}

\begin{tabular}{lll}
    Character & \\
    Obstacle & \\
    Powerups & \\
    DetectCollision & \\
    UpdateEnvironment & \\
    Score & \\
    DisplayObstacle & \\
    DisplayPowerups & \\
    DisplayEnvironment & \\
    DisplayWindow & \\
    DisplayCharacter & \\
    PlaySound & \\
    DisplayMenu & \\
    LoadAssets 
\end{tabular}

\subsection* {Syntax}

\subsubsection* {Exported Access Programs}

\begin{tabular}{| l | l | l | l |}
\hline
\textbf{Routine name} & \textbf{In} & \textbf{Out} & \textbf{Exceptions}\\
\hline
    GameController & --- & GameController & ---\\
\hline
    game\_loop & --- & --- & ---\\
\hline
    check\_user\_input & --- & --- & ---\\
\hline
\end{tabular}

\subsection* {Semantics}

\subsubsection* {State Variables}

game\_screen: pygame.display\\
obstacle\_list: seq of Obstacle\\
character : Character \\
powerup\_list : seq of Powerups\\
background\_music: pygame.mixer\\
sound\_effects: pygame.mixer\\
floor: pygame.image\\
floor\_position: $\mathbb{N}$\\
background: pygame.image\\
score\_count: Score


\subsubsection* {State Invariant}

None

\subsubsection* {Assumptions}

None

\subsubsection* {Access Routine Semantics}

GameController():
\begin{itemize}
    \item output: \textit{out} := \textit{self}
    \item transition: game\_screen := DisplayWindow.get\_game\_screen()\\
    obstacle\_list, powerup\_list := [], []\\
    character := Character(game\_screen, LoadAssets.load\_character()[0])\\
    background, sound\_effects := LoadAssets.load\_sound()\\
    floor := LoadAssets.load\_floor()\\
    floor\_position := 0\\
    background := LoadAsset.load\_background()\\
    score\_count := Score()
\end{itemize}

\noindent game\_loop():
\begin{itemize}
    \item transitions: It is the main game loop of the game that will continuously run until the current game session ends when the user presses the `Quit' button in the main menu. It will call methods from other modules to control the game play. The floor image will have its position constantly moving and being updated. The events of each keyboard input will be constantly monitored to ensure the user input is registered. The score will be incremented and depending on the current score, the background color will change. There will be random types of obstacles  and powerups generating at random times. There will be a constant check for collision detection between the character and obstacle, and character and powerups. If a collision between a character and obstacle has occured, the current game state stops and goes to the exits menu to compare current score and highest score achieved. 
\end{itemize}

\noindent check\_user\_input():
\begin{itemize}
    \item transition: Checks for user input and calls the corresponding method. The method is responsible for handling input for character control and moving the character based on the input. The method also handles the user inputs for starting, quitting, and going to the settings from the main menu. \\\\
\resizebox{15cm}{!}{
\begin{tabular}{|l|l|}
    \hline
    Input Key & Behaviour\\
    \hline
    \hline
    pygame.KEYDOWN & \\
    \hline
    pygame.QUIT & System exit\\
    pygame.K\_DOWN & character.duck() \& play\_duck\_sound()\\
    pygame.K\_UP & character.jump() \& play\_jump\_sound()\\
    pygame.K\_Up when jumping & character.double\_jump() \&
    play\_jump\_sound()\\
    pygame.K\_P & MenuController.pause\_menu(game\_screen)\\
    \hline
    \hline
    pygame.KEYUP & \\
    \hline
    pygame.K\_DOWN & character.stand()\\
    \hline
\end{tabular}
}
\end{itemize}

%%%%%%%%%%%%%%%%%%%%%%%%%%%%%%%%%%%%%%%%%%%%%%%%%%%%%%%%%
\newpage
%%%%%%%%%%%%%%%%%%%%%%%%%%%%%%%%%%%%%%%%%%%%%%%%%%%%%%%%%
\section*{Controller/MenuController}

\subsection* {Uses}

MainMenu

\subsection* {Syntax}

\subsubsection* {Exported Access Programs}

\begin{tabular}{| l | l | l | l |}
\hline
\textbf{Routine name} & \textbf{In} & \textbf{Out} & \textbf{Exceptions}\\
\hline
    setting\_menu & pygame.display & --- & ---\\
\hline
    pause\_menu & pygame.display & --- & ---\\
\hline
    exit\_menu & pygame.display & --- & ---\\
\hline
\end{tabular}

\subsection* {Semantics}

\subsubsection* {State Variables}

None

\subsubsection* {State Invariant}

None

\subsubsection* {Assumptions}

None

\subsubsection* {Access Routine Semantics}
\noindent setting\_menu(screen):
\begin{itemize}
    \item transition: Display the settings menu onto the screen and handle the volume and theme change according to user input. When the user press the `Back' button, the method terminates and control is shifted back to the game controller.
\end{itemize}

\noindent pause\_menu(screen):
\begin{itemize}
    \item transition: Display the pause menu onto the screen and freeze the current game state. The score count is kept unchanged and all the elements of the game (obstacles, character, and background) stop moving. If the `Resume' button is pressed, the game data current state is unfrozen, the method terminates and the control is shifted back to the game controller. If the Exit' button is pressed, the method terminates and the control is shifted back to the game controller (without saving the current game state upon exiting).
\end{itemize}

\noindent exit\_menu(screen):
\begin{itemize}
    \item transition: Display the exit menu onto the screen and restart or quit the game according to user input. When the user presses the `Exit' button, the game goes back to main menu. If the `Restart' button is pressed, a new game starts, the method terminates and control is passed back to the game controller whatever button is pressed.
\end{itemize}
\end {document}

